\documentclass[aip,jcp,reprint,floatfix]{revtex4-1}

\usepackage{amsmath,amssymb}
\usepackage{epsfig}
\usepackage{graphicx}
\usepackage{dcolumn}
\usepackage{bbold}
\usepackage{natbib}
\usepackage[normalem]{ulem} % to strike out plain text
\usepackage[colorlinks=true,citecolor={blue},urlcolor={blue}, linkcolor=blue]{hyperref}%
\usepackage[utf8]{inputenc}
\usepackage{amsthm}
\usepackage{braket}
\usepackage{verbatim}
\usepackage{mdframed}
\newtheorem{theorem}{Theorem}


\begin{document}


\title{Machine Learning and the Many-body Problem}

\author{Jane Kim}
\affiliation{National Superconducting Cyclotron Laboratory and Department of Physics and Astronomy, Michigan State University, East Lansing, MI 48824, USA}

\author{Vilde Moe Flugsrud}
\affiliation{Department of Physics, University of Oslo, N-0316 Oslo, Norway}

\author{Alfred Alocias Mariadason}
\affiliation{Department of Physics, University of Oslo, N-0316 Oslo, Norway}


\author{Morten Hjorth-Jensen}
\affiliation{National Superconducting Cyclotron Laboratory and Department of Physics and Astronomy, Michigan State University, East Lansing, MI 48824, USA}
\affiliation{Department of Physics and Center for Computing in Science Education, University of Oslo, N-0316 Oslo, Norway}



\begin{abstract}

\end{abstract}



\maketitle


\section{Introduction and motivation}

Traditional many-body methods like full configuration interaction
theory (FCI), coupled-cluster (CC) theory, many-body perturbation
theory (MBPT), in-medium similarity renormalization group (IMSRG),
various Monte Carlo methods and many other many-body approaches, have
been rather successful in describing properties of nuclear
systems. However, essentially all of these methods face what is
normally called the curse of dimensionality. For wave function based
method like FCI or CC theories where the original continuous problem
is discretized in terms selected basis functions and/or many-body
excitations, the dimensionality of the systems under study grows almost
exponentially when larger basis sets and/or number of particles are
included. Nuclear systems close to the limits of stability pose in
particular a tough problem to many-body practitioners in terms of an
increase of the number of relevant degrees of freedom. To describe say
weakly bound nuclei within the framework of a wave function based
approach, requires often a single-particle basis which includes bound,
weakly bound and unbound states, increasing thereby considerably the
number of many-body basis states. This renders often a standard
shell-model calculation infeasible. Within the above mentioned
many-body methods there are however several interesting theoretical
approachs which attempt at circumventing the dimensionality
curse. Smarter basis sets is one of these approaches, as well as the
resummation of specific correlations and the recently proposed
stochastic sampling of many-body states in both FCI and CC
calculations.

Recently, several authors of pointed to  possibilities within the
broad fields of Machine Learning and Quantum Computing as approaches
that hold great promise in tackling the ever increasing
dimensionalities. There are several groups worldwide which now focus
on Machine Learning and/or Quantum Computing applied to many-body
problems.

This work is inspired by the idea of representing the wave function with
a restricted Boltzmann machine (RBM), presented recently by \href{{http://science.sciencemag.org/content/355/6325/602}}{G. Carleo and M. Troyer, Science \textbf{355}, Issue 6325, pp. 602-606 (2017)}. They
named such a wave function/network a \emph{neural network quantum state} (NQS). In their article they apply it to the quantum mechanical
spin lattice systems of the Ising model and Heisenberg model, with
encouraging results.

Here we try to implement these algorithms to many-body systems that mimick a nuclear physics like problem, namely electrons confined to move in harmonic oscillator like potentials. 

Machine learning (ML) is an extremely rich field, in spite of its young age. The
increases we have seen during the last three decades in computational
capabilities have been followed by developments of methods and
techniques for analyzing and handling large date sets, relying heavily
on statistics, computer science and mathematics.  The field is rather
new and developing rapidly. 


Almost every problem in ML and data science starts with the same ingredients:
\begin{itemize}
\item The dataset $\mathbf{x}$ (could be some observable quantity of the system we are studying)

\item A model which is a function of a set of parameters $\mathbf{\alpha}$ that relates to the dataset, say a likelihood  function $p(\mathbf{x}\vert \mathbf{\alpha})$ or just a simple model $f(\mathbf{\alpha})$

\item A so-called \textbf{cost} function $\mathcal{C} (\mathbf{x}, f(\mathbf{\alpha}))$ which allows us to decide how well our model represents the dataset. 
\end{itemize}

\noindent
We seek to minimize the function $\mathcal{C} (\mathbf{x}, f(\mathbf{\alpha}))$ by finding the parameter values which minimize $\mathcal{C}$. This leads to  various minimization algorithms.

\section{What is Machine Learning}
Machine learning is the science of giving computers the ability to
learn without being explicitly programmed.  The idea is that there
exist generic algorithms which can be used to find patterns in a broad
class of data sets without having to write code specifically for each
problem. The algorithm will build its own logic based on the data.

Machine learning is a subfield of computer science, and is closely
related to computational statistics.  It evolved from the study of
pattern recognition in artificial intelligence (AI) research, and has
made contributions to AI tasks like computer vision, natural language
processing and speech recognition. It has also, especially in later
years, found applications in a wide variety of other areas, including
bioinformatics, economy, physics, finance and marketing.

You will notice however that many of the basic ideas discussed do come from Physics!

The approaches to machine learning are many, but are often split into two main categories. 
In \emph{supervised learning} we know the answer to a problem,
and let the computer deduce the logic behind it. On the other hand, \emph{unsupervised learning}
is a method for finding patterns and relationship in data sets without any prior knowledge of the system.
Some authours also operate with a third category, namely \emph{reinforcement learning}. This is a paradigm 
of learning inspired by behavioural psychology, where learning is achieved by trial-and-error, 
solely from rewards and punishment.

Another way to categorize machine learning tasks is to consider the desired output of a system.
Some of the most common tasks are:

\begin{itemize}
  \item Classification: Outputs are divided into two or more classes. The goal is to   produce a model that assigns inputs into one of these classes. An example is to identify  digits based on pictures of hand-written ones. Classification is typically supervised learning.

  \item Regression: Finding a functional relationship between an input data set and a reference data set.   The goal is to construct a function that maps input data to continuous output values.

  \item Clustering: Data are divided into groups with certain common traits, without knowing the different groups beforehand.  It is thus a form of unsupervised learning.
\end{itemize}


\begin{itemize}
\item An excellent reference, \href{{https://arxiv.org/abs/1803.08823}}{Mehta \emph{et al.}, arXiv:1803.08823 and Physics Reports in press (2018)}

\item A cute paper by \href{{https://journals.aps.org/prc/abstract/10.1103/PhysRevC.97.014306}}{Utama and Piekarewicz, Validating neural-network refinements of nuclear mass models, Phys. Rev. C 97, 014306}

\item \href{{https://journals.aps.org/prl/abstract/10.1103/PhysRevLett.120.156001}}{Every issue of Physical Review Letters has now one or more articles on ML}

\item \href{{https://github.com/CompPhysics/MachineLearning}}{Books and lectures notes} and see also the course \href{{https://www.uio.no/studier/emner/matnat/fys/FYS-STK4155/h18/index.html}}{FYS-STK3155/4155}


Here we will use so-called \textbf{reduced Boltzmann Machines} to simulate quantum many-body problems. For Monte Carlo aficionados, there is a very close similarity with what are called \textbf{shadow wave functions}, see the work of \href{{https://journals.aps.org/pre/abstract/10.1103/PhysRevE.90.053304}}{Pederiva and Kalos and collaborators, Phys Rev. E 90, 053304 (2014)}.






\subsection*{The system: two electrons in a harmonic oscillator trap in two dimensions}

The Hamiltonian of the quantum dot is given by
\[ \hat{H} = \hat{H}_0 + \hat{V}, 
\]
where $\hat{H}_0$ is the many-body HO Hamiltonian, and $\hat{V}$ is the
inter-electron Coulomb interactions. In dimensionless units,
\[ \hat{V}= \sum_{i < j}^N \frac{1}{r_{ij}},
\]
with $r_{ij}=\sqrt{\mathbf{r}_i^2 - \mathbf{r}_j^2}$.

This leads to the  separable Hamiltonian, with the relative motion part given by ($r_{ij}=r$)
\[ 
\hat{H}_r=-\nabla^2_r + \frac{1}{4}\omega^2r^2+ \frac{1}{r},
\]
plus a standard Harmonic Oscillator problem  for the center-of-mass motion.
This system has analytical solutions in two and three dimensions (\href{{https://journals.aps.org/pra/abstract/10.1103/PhysRevA.48.3561}}{M. Taut 1993 and 1994}). 

% !split
\subsection*{Quantum Monte Carlo Motivation}

% --- begin paragraph admon ---
\paragraph{}
Given a hamiltonian $H$ and a trial wave function $\Psi_T$, the variational principle states that the expectation value of $\langle H \rangle$, defined through 
\[
   \langle E \rangle =
   \frac{\int d\bm{R}\Psi^{\ast}_T(\bm{R})H(\bm{R})\Psi_T(\bm{R})}
        {\int d\bm{R}\Psi^{\ast}_T(\bm{R})\Psi_T(\bm{R})},
\]
is an upper bound to the ground state energy $E_0$ of the hamiltonian $H$, that is 
\[
    E_0 \le \langle H \rangle .
\]
In general, the integrals involved in the calculation of various  expectation values  are multi-dimensional ones. Traditional integration methods such as the Gauss-Legendre will not be adequate for say the  computation of the energy of a many-body system.
% --- end paragraph admon ---




\subsection*{Quantum Monte Carlo Motivation}


\subsection*{Quantum Monte Carlo Motivation}

% --- begin paragraph admon ---
\paragraph{Define a new quantity.}
\[
   E_L(\bm{R},\bm{\alpha})=\frac{1}{\psi_T(\bm{R},\bm{\alpha})}H\psi_T(\bm{R},\bm{\alpha}),
\]
called the local energy, which, together with our trial PDF yields
\[
  E[\bm{\alpha}]=\int P(\bm{R})E_L(\bm{R},\bm{\alpha}) d\bm{R}\approx \frac{1}{N}\sum_{i=1}^NE_L(\bm{R_i},\bm{\alpha})
\]
with $N$ being the number of Monte Carlo samples.
% --- end paragraph admon ---




% !split
\subsection*{Quantum Monte Carlo}

% --- begin paragraph admon ---
\paragraph{}
The Algorithm for performing a variational Monte Carlo calculations runs thus as this

\begin{itemize}
   \item Initialisation: Fix the number of Monte Carlo steps. Choose an initial $\bm{R}$ and variational parameters $\alpha$ and calculate $\left|\psi_T(\bm{R},\bm{\alpha})\right|^2$. 

   \item Initialise the energy and the variance and start the Monte Carlo calculation by looping over trials.
\begin{itemize}

      \item Calculate  a trial position  $\bm{R}_p=\bm{R}+r*step$ where $r$ is a random variable $r \in [0,1]$.

      \item Metropolis algorithm to accept or reject this move  $w = P(\bm{R}_p,\bm{\alpha})/P(\bm{R},\bm{\alpha})$.

      \item If the step is accepted, then we set $\bm{R}=\bm{R}_p$. 

      \item Update averages

\end{itemize}

\noindent
   \item Finish and compute final averages.
\end{itemize}

\noindent
Observe that the jumping in space is governed by the variable \emph{step}. This is often called brute-force sampling.
Need importance sampling to get more relevant sampling.
% --- end paragraph admon ---



% !split
\subsection*{The trial wave function}

% --- begin paragraph admon ---
\paragraph{}
We want to perform  a Variational Monte Carlo calculation of the ground state of two electrons in a quantum dot well with different oscillator energies, assuming total spin $S=0$.
Our trial wave function has the following form
\begin{equation}
   \psi_{T}(\bm{r}_1,\bm{r}_2) = 
   C\exp{\left(-\alpha_1\omega(r_1^2+r_2^2)/2\right)}
   \exp{\left(\frac{r_{12}}{(1+\alpha_2 r_{12})}\right)}, 
\label{eq:trial}
\end{equation}
where the $\alpha$s represent our variational parameters, two in this case.

Why does the trial function look like this? How did we get there? \textbf{This will be our main motivation} for switching to
Machine Learning.
% --- end paragraph admon ---



% !split
\subsection*{The correlation part of the wave function}

To find an ansatz for the correlated part of the wave function, it is useful to rewrite the two-particle
local energy in terms of the relative and center-of-mass motion.
Let us denote the distance between the two electrons as
$r_{12}$. We omit the center-of-mass motion since we are only interested in the case when
$r_{12} \rightarrow 0$. The contribution from the center-of-mass (CoM) variable $\bm{R}_{\mathrm{CoM}}$
gives only a finite contribution.
We focus only on the terms that are relevant for $r_{12}$ and for three dimensions. The relevant local energy becomes then
\[
\lim_{r_{12} \rightarrow 0}E_L(R)=
    \frac{1}{{\cal R}_T(r_{12})}\left(2\frac{d^2}{dr_{ij}^2}+\frac{4}{r_{ij}}\frac{d}{dr_{ij}}+
\frac{2}{r_{ij}}-\frac{l(l+1)}{r_{ij}^2}+2E
\right){\cal R}_T(r_{12}) = 0.
\]
Set $l=0$ and we have the so-called \textbf{cusp} condition
\[
\frac{d {\cal R}_T(r_{12})}{dr_{12}} = -\frac{1}{2(l+1)}
{\cal R}_T(r_{12})\qquad r_{12}\to 0
\]

% !split
\subsection*{Resulting ansatz}
The above  results in
\[
{\cal R}_T  \propto \exp{(r_{ij}/2)}, 
\]
for anti-parallel spins and 
\[
{\cal R}_T  \propto \exp{(r_{ij}/4)}, 
\]
for anti-parallel spins. 
This is the so-called cusp condition for the relative motion, resulting in a minimal requirement
for the correlation part of the wave fuction.
For general systems containing more than say two electrons, we have this
condition for each electron pair $ij$.






% --- begin paragraph admon ---
\paragraph{}
The elements of the gradient of the local energy are then (using the chain rule and the hermiticity of the Hamiltonian)
\[
\bar{E}_{i}= 2\left( \langle \frac{\bar{\Psi}_{i}}{\Psi}E_L\rangle -\langle \frac{\bar{\Psi}_{i}}{\Psi}\rangle\langle E_L \rangle\right).
\]
From a computational point of view it means that you need to compute the expectation values of 
\[
\langle \frac{\bar{\Psi}_{i}}{\Psi}E_L\rangle,
\]
and
\[
\langle \frac{\bar{\Psi}_{i}}{\Psi}\rangle\langle E_L\rangle
\]
These integrals are evaluted using MC intergration (with all its possible error sources). 
We can then use methods like stochastic gradient or other minimization methods to find the optimal variational parameters (I don't discuss this topic here, but these methods are very important in ML).
% --- end paragraph admon ---



% !split
\subsection*{How do we define our cost function?}

% --- begin paragraph admon ---
\paragraph{}
We have a model, our likelihood function. 

How should we define the cost function?


% !split
\subsection*{Why Boltzmann machines?}

What is known as restricted Boltzmann Machines (RMB) have received a lot of attention lately. 
One of the major reasons is that they can be stacked layer-wise to build deep neural networks that capture complicated statistics.

The original RBMs had just one visible layer and a hidden layer, but recently so-called Gaussian-binary RBMs have gained quite some popularity in imaging since they are capable of modeling continuous data that are common to natural images. 

Furthermore, they have been used to solve complicated quantum mechanical many-particle problems or classical statistical physics problems like the Ising and Potts classes of models. 



% !split
\subsection*{Boltzmann Machines}

Why use a generative model rather than the more well known discriminative deep neural networks (DNN)? 

\begin{itemize}
\item Discriminitave methods have several limitations: They are mainly supervised learning methods, thus requiring labeled data. And there are tasks they cannot accomplish, like drawing new examples from an unknown probability distribution.

\item A generative model can learn to represent and sample from a probability distribution. The core idea is to learn a parametric model of the probability distribution from which the training data was drawn. As an example
\begin{enumerate}

 \item A model for images could learn to draw new examples of cats and dogs, given a training dataset of images of cats and dogs.

 \item Generate a sample of an ordered or disordered Ising model phase, having been given samples of such phases.

 \item Model the trial function for Monte Carlo calculations
\end{enumerate}

\noindent
\end{itemize}

\noindent
% !split
\subsection*{Some similarities and differences from DNNs}

\begin{enumerate}
\item Both use gradient-descent based learning procedures for minimizing cost functions

\item Energy based models don't use backpropagation and automatic differentiation for computing gradients, instead turning to Markov Chain Monte Carlo methods.

\item DNNs often have several hidden layers. A restricted Boltzmann machine has only one hidden layer, however several RBMs can be stacked to make up Deep Belief Networks, of which they constitute the building blocks.
\end{enumerate}

\noindent
History: The RBM was developed by amongst others Geoffrey Hinton, called by some the "Godfather of Deep Learning", working with the University of Toronto and Google.


% !split
\subsection*{Boltzmann machines (BM)}


% --- begin paragraph admon ---
\paragraph{}
A BM is what we would call an undirected probabilistic graphical model
with stochastic continuous or discrete units.
% --- end paragraph admon ---



% --- begin paragraph admon ---
\paragraph{}
It is interpreted as a stochastic recurrent neural network where the
state of each unit(neurons/nodes) depends on the units it is connected
to. The weights in the network represent thus the strength of the
interaction between various units/nodes.
% --- end paragraph admon ---



% --- begin paragraph admon ---
\paragraph{}
It turns into a Hopfield network if we choose deterministic rather
than stochastic units. In contrast to a Hopfield network, a BM is a
so-called generative model. It allows us to generate new samples from
the learned distribution.
% --- end paragraph admon ---



% !split
\subsection*{A standard BM setup}


% --- begin paragraph admon ---
\paragraph{}
A standard BM network is divided into a set of observable and visible units $\hat{x}$ and a set of unknown hidden units/nodes $\hat{h}$.
% --- end paragraph admon ---




% --- begin paragraph admon ---
\paragraph{}
Additionally there can be bias nodes for the hidden and visible layers. These biases are normally set to $1$.
% --- end paragraph admon ---




% --- begin paragraph admon ---
\paragraph{}
BMs are stackable, meaning they cwe can train a BM which serves as input to another BM. We can construct deep networks for learning complex PDFs. The layers can be trained one after another, a feature which makes them popular in deep learning
% --- end paragraph admon ---



However, they are often hard to train. This leads to the introduction of so-called restricted BMs, or RBMS.
Here we take away all lateral connections between nodes in the visible layer as well as connections between nodes in the hidden layer. The network is illustrated in the figure below.



% !split
\subsection*{The structure of the RBM network}



\vspace{6mm}

% inline figure
\centerline{\includegraphics[width=1.0\linewidth]{figures/RBM.pdf}}

\vspace{6mm}





% !split
\subsection*{The network}

\textbf{The network layers}:
\begin{enumerate}
 \item A function $\mathbf{x}$ that represents the visible layer, a vector of $M$ elements (nodes). This layer represents both what the RBM might be given as training input, and what we want it to be able to reconstruct. This might for example be the pixels of an image, the spin values of the Ising model, or coefficients representing speech.

 \item The function $\mathbf{h}$ represents the hidden, or latent, layer. A vector of $N$ elements (nodes). Also called "feature detectors".
\end{enumerate}

\noindent
% !split
\subsection*{Goals}

The goal of the hidden layer is to increase the model's expressive power. We encode complex interactions between visible variables by introducing additional, hidden variables that interact with visible degrees of freedom in a simple manner, yet still reproduce the complex correlations between visible degrees in the data once marginalized over (integrated out).

Examples of this trick being employed in physics: 
\begin{enumerate}
 \item The Hubbard-Stratonovich transformation

 \item The introduction of ghost fields in gauge theory

 \item Shadow wave functions in Quantum Monte Carlo simulations
\end{enumerate}

\noindent
\textbf{The network parameters, to be optimized/learned}:
\begin{enumerate}
 \item $\mathbf{a}$ represents the visible bias, a vector of same length as $\mathbf{x}$.

 \item $\mathbf{b}$ represents the hidden bias, a vector of same lenght as $\mathbf{h}$.

 \item $W$ represents the interaction weights, a matrix of size $M\times N$.
\end{enumerate}

\noindent
% !split
\subsection*{Joint distribution}

The restricted Boltzmann machine is described by a Boltzmann distribution
\begin{align}
	P_{rbm}(\mathbf{x},\mathbf{h}) = \frac{1}{Z} e^{-\frac{1}{T_0}E(\mathbf{x},\mathbf{h})},
\end{align}
where $Z$ is the normalization constant or partition function, defined as 
\begin{align}
	Z = \int \int e^{-\frac{1}{T_0}E(\mathbf{x},\mathbf{h})} d\mathbf{x} d\mathbf{h}.
\end{align}
It is common to ignore $T_0$ by setting it to one. 




% !split
\subsection*{Network Elements, the energy function}

The function $E(\mathbf{x},\mathbf{h})$ gives the \textbf{energy} of a
configuration (pair of vectors) $(\mathbf{x}, \mathbf{h})$. The lower
the energy of a configuration, the higher the probability of it. This
function also depends on the parameters $\mathbf{a}$, $\mathbf{b}$ and
$W$. Thus, when we adjust them during the learning procedure, we are
adjusting the energy function to best fit our problem.

An expression for the energy function is
\[
E(\hat{x},\hat{h}) = -\sum_{ia}^{NA}b_i^a \alpha_i^a(x_i)-\sum_{jd}^{MD}c_j^d \beta_j^d(h_j)-\sum_{ijad}^{NAMD}b_i^a \alpha_i^a(x_i)c_j^d \beta_j^d(h_j)w_{ij}^{ad}.
\]

Here $\beta_j^d(h_j)$ and $\alpha_i^a(x_j)$ are so-called transfer functions that map a given input value to a desired feature value. The labels $a$ and $d$ denote that there can be multiple transfer functions per variable. The first sum depends only on the visible units. The second on the hidden ones. \textbf{Note} that there is no connection between nodes in a layer.

The quantities $b$ and $c$ can be interpreted as the visible and hidden biases, respectively.

The connection between the nodes in the two layers is given by the weights $w_{ij}$. 

% !split
\subsection*{Defining different types of RBMs}
There are different variants of RBMs, and the differences lie in the types of visible and hidden units we choose as well as in the implementation of the energy function $E(\mathbf{x},\mathbf{h})$. 


% --- begin paragraph admon ---
\paragraph{Binary-Binary RBM:}

RBMs were first developed using binary units in both the visible and hidden layer. The corresponding energy function is defined as follows:
\begin{align}
	E(\mathbf{x}, \mathbf{h}) = - \sum_i^M x_i a_i- \sum_j^N b_j h_j - \sum_{i,j}^{M,N} x_i w_{ij} h_j,
\end{align}
where the binary values taken on by the nodes are most commonly 0 and 1.
% --- end paragraph admon ---




\paragraph{Gaussian-Binary RBM:}

Another varient is the RBM where the visible units are Gaussian while the hidden units remain binary:
\begin{align}
	E(\mathbf{x}, \mathbf{h}) = \sum_i^M \frac{(x_i - a_i)^2}{2\sigma_i^2} - \sum_j^N b_j h_j - \sum_{i,j}^{M,N} \frac{x_i w_{ij} h_j}{\sigma_i^2}. 
\end{align}

\subsection*{Cost function}

When working with a training dataset, the most common training approach is maximizing the log-likelihood of the training data. The log likelihood characterizes the log-probability of generating the observed data using our generative model. Using this method our cost function is chosen as the negative log-likelihood. The learning then consists of trying to find parameters that maximize the probability of the dataset, and is known as Maximum Likelihood Estimation (MLE).
Denoting the parameters as $\bm{\theta} = a_1,...,a_M,b_1,...,b_N,w_{11},...,w_{MN}$, the log-likelihood is given by
\begin{align}
	\mathcal{L}(\{ \theta_i \}) &= \langle \text{log} P_\theta(\bm{x}) \rangle_{data} \\
	&= - \langle E(\bm{x}; \{ \theta_i\}) \rangle_{data} - \text{log} Z(\{ \theta_i\}),
\end{align}
where we used that the normalization constant does not depend on the data, $\langle \text{log} Z(\{ \theta_i\}) \rangle = \text{log} Z(\{ \theta_i\})$
Our cost function is the negative log-likelihood, $\mathcal{C}(\{ \theta_i \}) = - \mathcal{L}(\{ \theta_i \})$

\subsection*{RBMs for the quantum many body problem}

The idea of applying RBMs to quantum many body problems was presented by G. Carleo and M. Troyer, working with ETH Zurich and Microsoft Research.

Some of their motivation included

\begin{itemize}
\item "The wave function $\Psi$ is a monolithic mathematical quantity that contains all the information on a quantum state, be it a single particle or a complex molecule. In principle, an exponential amount of information is needed to fully encode a generic many-body quantum state."

\item There are still interesting open problems, including fundamental questions ranging from the dynamical properties of high-dimensional systems to the exact ground-state properties of strongly interacting fermions.

\item The difficulty lies in finding a general strategy to reduce the exponential complexity of the full many-body wave function down to its most essential features. That is
\begin{enumerate}

\item $\rightarrow$ Dimensional reduction

\item $\rightarrow$ Feature extraction

\end{enumerate}

\noindent
\item Among the most successful techniques to attack these challenges, artifical neural networks play a prominent role.

\item Want to understand whether an artifical neural network may adapt to describe a quantum system.
\end{itemize}

\noindent
% !split
\subsection*{Choose the right RBM}

Carleo and Troyer applied the RBM to the quantum mechanical spin lattice systems of the Ising model and Heisenberg model, with encouraging results. Our goal is to test the method on systems of moving particles. For the spin lattice systems it was natural to use a binary-binary RBM, with the nodes taking values of 1 and -1. For moving particles, on the other hand, we want the visible nodes to be continuous, representing position coordinates. Thus, we start by choosing a Gaussian-binary RBM, where the visible nodes are continuous and hidden nodes take on values of 0 or 1. If eventually we would like the hidden nodes to be continuous as well the rectified linear units seem like the most relevant choice.

% !split
\subsection*{Representing the wave function}
The wavefunction should be a probability amplitude depending on $\bm{x}$. The RBM model is given by the joint distribution of $\bm{x}$ and $\bm{h}$
\begin{align}
	F_{rbm}(\mathbf{x},\mathbf{h}) = \frac{1}{Z} e^{-\frac{1}{T_0}E(\mathbf{x},\mathbf{h})}.
\end{align}
To find the marginal distribution of $\bm{x}$ we set:
\begin{align}
	F_{rbm}(\mathbf{x}) &= \sum_\mathbf{h} F_{rbm}(\mathbf{x}, \mathbf{h}) \\
				&= \frac{1}{Z}\sum_\mathbf{h} e^{-E(\mathbf{x}, \mathbf{h})}.
\end{align}
Now this is what we use to represent the wave function, calling it a neural-network quantum state (NQS)
\begin{align}
	\Psi (\mathbf{X}) &= F_{rbm}(\mathbf{x}) \\
	&= \frac{1}{Z}\sum_{\bm{h}} e^{-E(\mathbf{x}, \mathbf{h})} \\
	&= \frac{1}{Z} \sum_{\{h_j\}} e^{-\sum_i^M \frac{(x_i - a_i)^2}{2\sigma^2} + \sum_j^N b_j h_j + \sum_{i,j}^{M,N} \frac{x_i w_{ij} h_j}{\sigma^2}} \\
	&= \frac{1}{Z} e^{-\sum_i^M \frac{(x_i - a_i)^2}{2\sigma^2}} \prod_j^N (1 + e^{b_j + \sum_i^M \frac{x_i w_{ij}}{\sigma^2}}). \\
\end{align}

% !split
\subsection*{Choose the cost function}
Now we don't necessarily have training data (unless we generate it by using some other method). However, what we do have is the variational principle which allows us to obtain the ground state wave function by minimizing the expectation value of the energy of a trial wavefunction (corresponding to the untrained NQS). Similarly to the traditional variational Monte Carlo method then, it is the local energy we wish to minimize. The gradient to use for the stochastic gradient descent procedure is
\begin{align}
	G_i = \frac{\partial \langle E_L \rangle}{\partial \theta_i}
	= 2(\langle E_L \frac{1}{\Psi}\frac{\partial \Psi}{\partial \theta_i} \rangle - \langle E_L \rangle \langle \frac{1}{\Psi}\frac{\partial \Psi}{\partial \theta_i} \rangle ),
\end{align}
where the local energy is given by
\begin{align}
	E_L = \frac{1}{\Psi} \hat{\mathbf{H}} \Psi.
\end{align}



You can find the codes for the simple two-electron case at the Github repository \href{{https://github.com/mhjensenseminars/MachineLearningTalk/tree/master/doc/Programs/MLcpp/src}}{\nolinkurl{https://github.com/mhjensenseminars/MachineLearningTalk/tree/master/doc/Programs/MLcpp/src}}. Python codes to come, only c++ as of now. 

The trial wave function is based on the product of a Slater determinant with Gaussian orbitals, a simple Jastrow factor $\exp{(r_{ij})}$ and the reduced Boltzmann machines. 

The Broyden-Fletcher-Goldfarb-Shanno algorithm was used for the minimization. We used $14$ hidden nodes in the calculations below.

\centerline{\includegraphics[width=0.9\linewidth]{figures/figN2.pdf}}

\centerline{\includegraphics[width=0.9\linewidth]{figures/figN6.pdf}}


% !split
\subsection*{Conclusions and where do we stand}

% --- begin paragraph admon ---
\paragraph{}
\begin{itemize}
\item A simple extension of the work of \href{{http://science.sciencemag.org/content/355/6325/602}}{G. Carleo and M. Troyer, Science \textbf{355}, Issue 6325, pp. 602-606 (2017)} gives excellent results for two-electron systems as well as good agreement with standard VMC calculations for $N=6$ and $N=12$ electrons.

\item Minimization problem can be tricky.

\item Anti-symmetry dealt with multiplying the trail wave function with an optimized Slater determinant.

\item To come: Analysis of wave function from ML and compare with diffusion and Variational Monte Carlo calculations as well as the analytical results of Taut for the two-electron case.

\item Extend to more fermions. How do we deal with the antisymmetry of the multi-fermion wave function?
\begin{enumerate}

 \item Here we used standard Hartree-Fock theory to define an optimal Slater determinant. Takes care of the antisymmetry. What about constructing an anti-symmetrized network function?

 \item Use thereafter ML to determine the correlated part of the wafe function (including a standard Jastrow factor).

 \item Test this for multi-fermion systems and compare with other many-body methods.

\end{enumerate}

\noindent
\item Can we use ML to find out which correlations are relevant and thereby diminish the dimensionality problem in say CC or SRG theories?
\end{itemize}


\end{document}

