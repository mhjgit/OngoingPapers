\documentclass[aip,jcp,reprint,floatfix]{revtex4-1}

\usepackage{amsmath,amssymb}
\usepackage{epsfig}
\usepackage{graphicx}
\usepackage{dcolumn}
\usepackage{bbold}
\usepackage{natbib}
\usepackage[normalem]{ulem} % to strike out plain text
\usepackage[colorlinks=true,citecolor={blue},urlcolor={blue}, linkcolor=blue]{hyperref}%
\usepackage[utf8]{inputenc}
\usepackage{amsthm}
\usepackage{braket}
\usepackage{verbatim}
\usepackage{mdframed}
\newtheorem{theorem}{Theorem}


\begin{document}


\title{Machine Learning and the Many-body Problem}

\author{Jane Kim}
\affiliation{National Superconducting Cyclotron Laboratory and Department of Physics and Astronomy, Michigan State University, East Lansing, MI 48824, USA}
\affiliation{Department of Physics, University of Oslo, N-0316 Oslo, Norway}

\author{Vilde Flugsrud}
\affiliation{Department of Physics, University of Oslo, N-0316 Oslo, Norway}

\author{Sebastian G.~Winther-Larsen}
\affiliation{Department of Physics, University of Oslo, N-0316 Oslo, Norway}


\author{Morten Hjorth-Jensen}
\affiliation{National Superconducting Cyclotron Laboratory and Department of Physics and Astronomy, Michigan State University, East Lansing, MI 48824, USA}
\affiliation{Department of Physics and Center for Computing in Science Education, University of Oslo, N-0316 Oslo, Norway}



\begin{abstract}

\end{abstract}

\pacs{32.80.Rm, 31.15.A-, 42.65.Ky}

\maketitle




\section{Introduction}

Quantum mechanics is a theory that describes/models the properties of
microscopic systems. It postulates that given the so-called
wavefunction, $\Psi(\mathbf{r},t)$, we can in principle compute all
that there is to know about the system.  Given suitable initial
conditions $\Psi(\mathbf{r},t)$ can be determined for all future time
by solving the time-dependent Schrödinger equation (TDSE),
\begin{equation}
 i \hbar \frac{\partial }{\partial t} \Psi(\mathbf{r},t) = \hat{H} \Psi(\mathbf{r},t).
\end{equation}
The initial condition is typically taken as a linear combination of what is called stationary states, which in turn can be found by solving what is known as the 
time-independent Schrödinger (TISE) equation
\begin{equation}
 \hat{H} \Psi(\mathbf{r}) = E \Psi(\mathbf{r}).
\end{equation}
Exact/analytical solutions to the TISE is possible only for the
simplest systems and general solutions to the TDSE is even
rarer. Thus, one must resort to approximative/numerical methods. Of
particular interest is the so-called many-body problem where one
considers systems of a large number of interacting particles, where
large can be anywhere from two to infinity.

Several approaches for solving the many-body TISE
numerically/approximatively has been devised such as Hartree-Fock
(HF), Density Functional Theory (DFT), Configuration Interaction (CI)
and Coupled Cluster (CC). While efficient, Hartree-Fock and DFT are
insufficient if one wants a high degree of accuracy. Configuration
Interaction and Coupled Cluster methods are hierarchical in the sense
that one can systematically construct increasly accurate
approximations. If the CI method is not truncated we have what is
known as Full Configuration Interaction (FCI). FCI can be seen as
exact (within some finite space), however it suffers from exponential
scaling. Truncated CI methods which would achieve polynomial scaling
are problematic since they are not size-consistent and
extensive. Truncated CC methods one the other hand are size-consistent
and extensive and achieves polynomial scaling. Due to this fact
Coupled Cluster is considered the gold standard of many-body
techniques if high accuracy is desired.

Similarly, there exists different approximations to the solution of
the TDSE with the Multiconfiguration Time-Dependent Hartree-Fock
method (MCTDHF) being considered the most accurate. MCTDHF, which is
combination of CI and HF generalized to the time domain, suffers from
exponential scaling. In a recent article \cite{Kvaal12}, Simen Kvaal
proposes a method based on Coupled-Cluster theory, the so-called
Orbital Adaptive Time Dependent Coupled Cluster method (OATDCC), which
is a hierarchical approximation to the MCTDHF method. The OATDCC
inherits size-consisteny and extensivity from the CC method and
achieves polynomial scaling.

\section{Theory/Formalism}
\subsection{The model Hamiltonian}
We shall model the quantum dot systems as a collection of $N$
nonrelativistic electrons of mass $m$ in $d$-dimensional space trapped
by a confining potential $v_{\text{conf}}(\mathbf{\hat{r}})$. Here
$\hat{\mathbf{r}}$ is the position operator in $d$-dimensional space
\begin{equation}
 \hat{\mathbf{r}} \equiv \sum_{i=1}^d \hat{\mathbf{x}}_i \hat{\mathbf{e}}_i
\end{equation}
with $\hat{\mathbf{e}}_i$ a unit vector.

The electrons interact through the Coulomb interaction $e^2/(4\pi
\epsilon R)$, where $e$ is the electron charge, $\epsilon$ the
permitivity of the medium and $R$ is the distance between two
interacting electrons.

For convenience, we will use atomic untis, chosen such that $\hbar
\equiv m \equiv e \equiv 4\pi \epsilon \equiv 1$. Thus, all energies
are presented in hartrees per $\hbar$.

In atomic units, the many-body problem is described by the Hamiltonian
on second quantized form
\begin{align}
 \hat{H} = \hat{H}_1 + \hat{H}_2 = &\sum_{pq} \braket{p|\hat{h}_0|q}\hat{a}_p^\dagger\hat{a}_q \nonumber \\
 + \frac{1}{4}&\sum_{pqrs}\braket{pq|\hat{u}|rs}_{\text{AS}}\hat{a}_p^\dagger\hat{a}_q^\dagger\hat{a}_s\hat{a}_r \label{FieldFreeHamiltonian}
\end{align}
where $\hat{H}_1$ and $\hat{H}_2$ are its one- and two-body parts. 

The operator $\hat{h}_0$ is the single-particle Hamiltonian
\begin{equation}
 \hat{h}_0 = -\frac{1}{2}\nabla^2 + v_{\text{conf}}(\mathbf{\hat{r}})
\end{equation}
while $\hat{u}$ represents the Coulomb interaction. Here $\ket{p}$ is
a single-particle basis function which in its position representation
is given by
\begin{equation}
 \braket{\mathbf{r}|p} \equiv \phi_p(\mathbf{r}).
\end{equation}
In practice, we have to choose some single-particle basis to compute
within. One typical choice is to take the single-particle basis to be
the eigenfunctions of $\hat{h}_0$ such that
\begin{equation}
 \hat{h}_0 \ket{p} = \epsilon_p \ket{p},
\end{equation}
where $\epsilon_p$ is the corresponding eigenvalue.

The quantities $\braket{p|\hat{h}_0|q}$ and
$\braket{pq|\hat{u}|rs}_{\text{AS}}$ are referred to as \textit{matrix
  elements} and must be computed relative to the chosen
single-particle basis. In particular,
$\braket{pq|\hat{u}|rs}_{\text{AS}}$ are \textit{antisymmetrized}
matrix elements, defined by
\begin{equation}
 \braket{pq|\hat{u}|rs}_{\text{AS}} \equiv u^{pq}_{rs} - u^{pq}_{sr}
\end{equation}
where $u^{pq}_{rs}$ is the interaction integral, 
\begin{equation}
 u^{pq}_{rs} \equiv \int \phi^*_p(\mathbf{r}_1) \phi^*_q(\mathbf{r}_2) \hat{u}(\mathbf{r}_1,\mathbf{r}_2) \phi_r(\mathbf{r}_1)\phi_s(\mathbf{r}_2) d\mathbf{r}_1 d\mathbf{r}_2.
\end{equation}
In the rest of this paper we will drop the subscript AS for the
antisymmetrized matrix elements.

In order to generate dynamics the electrons are coupled to a time
dependent monochromatic laser field, with frequency $\omega$ and
amplitude $\mathcal{E}_0$, polarized along the $x$-axis
\cite{Zanghellini04}
\begin{equation}
 v_{\text{laser}}(\mathbf{\hat{r},t}) = \hat{\mathbf{x}}_1 \mathcal{E}_0 \sin(\omega t),
\end{equation}
resulting in a time dependent single-particle Hamiltonian
\begin{equation}
 \hat{h}_t = \hat{h}_0 + v_{\text{laser}}(\mathbf{\hat{r},t}).
\end{equation}
The choice of polarization axis is arbitrary. 

The total time dependent Hamiltonian of the system is then given by
\begin{align}
 \hat{H}(t) = &\sum_{pq} \braket{p|\hat{h}_t|q}\hat{a}_p^\dagger\hat{a}_q \nonumber \\
 + \frac{1}{4}&\sum_{pqrs}\braket{pq|\hat{u}|rs}_{\text{AS}}\hat{a}_p^\dagger\hat{a}_q^\dagger\hat{a}_s\hat{a}_r.
\end{align}
It is important to note that this implies that we have to update the
matrix elements $\braket{p|\hat{h}_t|q}$ during the
simulation. However, in this work we consider only methods where the
single-particle basis is held fixed in time, hence the two-body matrix
elements can be computed $\braket{pq|\hat{u}|rs}$ once and for all.

We will handle the time dependent problem in two steps. First we
prepare a system in its groundstate by solving the time independent
Schrödinger equation (TISE) for the field-free Hamiltonian
\eqref{FieldFreeHamiltonian}.  Then we apply the time dependent laser
field and compute the time evolution from the time dependent
Schrödinger equation (TDSE).  Before we discuss the metods for solving
the TISE and TDSE respectively we will adress the choice of
single-particle basis.

\subsection{Choice of single-particle basis}
The choice of single-particle basis will in general depend on the
problem. In quantum chemistry it is customary to expand the
eigenfunctions, $\{ \psi_k \}$, of $\hat{h}_0$ in (non-orthogonal)
Gaussian basis functions $g_p(\mathbf{r})$
\begin{equation}
 \hat{h}_0 \ket{\psi_k} = \hat{h}_0 \sum_{p=1}^L g_p(\mathbf{r}) = \epsilon_k \sum_{p=1}^L g_p(\mathbf{r})
\end{equation}
resulting in a generalized eigenvalue problem which must be solved for
the eigenfunctions. Ideally we would let $L \rightarrow \infty$ but
must be taken as finite in practice.  The choice of Gaussian basis
functions is popular due to the fact that matrix elemtents can be
evaluated efficiently \cite{Helgaker00book}.

Another popular choice, which is the one chosen in this work, is to
use the eigenfunctions of the Harmonic oscillator (HO)
\begin{equation}
 \left(-\frac{1}{2}\nabla^2 + \frac{1}{2} \omega_{\text{HO}}\hat{\mathbf{r}}^2 \right) \phi_k(\mathbf{r}) = \epsilon_k \phi_k(\mathbf{r})  
\end{equation}
as a single-particle basis. The solutions to the HO-problem is well
known and the eigenfunctions have closed form expressions.

Of particular importance for our purposes is the fact that the
interaction integrals
\begin{equation}
 u^{pq}_{rs} \equiv \int \phi^*_p(\mathbf{r}_1) \phi^*_q(\mathbf{r}_2) \hat{u}(\mathbf{r}_1,\mathbf{r}_2) \phi_r(\mathbf{r}_1)\phi_s(\mathbf{r}_2) d\mathbf{r}_1 d\mathbf{r}_2
\end{equation}
can be solved analytically in two- and three-dimensions, using
HO-basis functions in polar and spherical coordinates respectively
\cite{Anisimovas98, Vorrath2003}. If the confining potential is chosen
to be the Harmonic oscillator potential we are essentially done and
can move on to the many-body treatment of the problem.

Additionally we may approximate the eigenfunctions of more general
confining potentials by expanding the solutions in the HO-functions.

Consider the general problem
\begin{equation}
 \hat{h}_0 \ket{\psi_k} = \epsilon_k \ket{\psi_k} \label{EigenfuncEq}
\end{equation}
where 
\begin{equation}
 \hat{h}_0 = -\frac{1}{2}\nabla^2 + v_{\text{conf}}(\mathbf{r}).
\end{equation}
Expand the eigenfunctions $\ket{\psi_k}$ in the HO-functions $\ket{\phi_p}$
\begin{equation}
 \ket{\psi_k} = \sum_{p=1}^L c_{p,k} \ket{\phi_p} 
\end{equation}
and insert into Eq.\eqref{EigenfuncEq}. Left-projection with
$\bra{\phi_q}$ results in the following expression
\begin{equation}
 \sum_{p=1}^L c_{p,k} \braket{\phi_q|\hat{h}_0|\phi_p} = \epsilon_k \sum_{p=1}^L c_{p,k} \delta_{q,p}
\end{equation}
where we have used the orthogonality of the HO-functions. This can be written on matrix form as 
\begin{equation}
 HC = C\mathcal{E}
\end{equation}
where $H_{qp} \equiv \braket{\phi_q|\hat{h}_0|\phi_p}$, $\mathcal{E}$
is a diagonal matrix containing the eigenvalues on its diagonal, and
$C$ is matrix conatining the expansions coefficients for basis
function $k$ in its columns. The eigenproblem must be solved for the
expansions coefficients, resulting in a approximation of the
interaction integrals in terms of the known intercation integrals
$\braket{\phi_\alpha \phi_\beta|\hat{u}|\phi_\gamma \phi_\delta}$
given by
\begin{equation}
 \braket{\psi_p \psi_q| \hat{u}| \psi_r \psi_s} = \sum_{\alpha \beta \gamma \delta} c^*_{p,\alpha} c^*_{q,\beta} c^*_{r,\gamma} c^*_{s,\delta} \braket{\phi_\alpha \phi_\beta|\hat{u}|\phi_\gamma \phi_\delta}.
\end{equation}

The quality of such an approximation will naturally depend on how
close the confining potential is to the Harmonic oscillator
potential. Nevertheless it constitutes a convenient starting point for
investigations of more general potentials.

\subsection{Configuration interaction theory}

\subsubsection{Time independent Configuration interaction}
The conceptually simplest method for solving the many-problem is
probably that of Configuration interaction (CI).

Let us first consider the time independent Schrödinger equation 
\begin{equation}
 \hat{H} \ket{\Psi_k} = E_k \ket{\Psi_k}
\end{equation}
where $\hat{H}$ is the field-free Hamiltonian given by
Eq. \eqref{FieldFreeHamiltonian}, $\ket{\Psi_k}$ is the many-body
eigenfunction and $E_k$ its corresponding eigenvalue. It is customary
to use a (finite in practice) basis of orthogonal Slater determinants
(SD), $\{ \Phi_p \}_{p=1}^{N_{\text{sd}}}$ where $N_{\text{sd}}$ is
the number of determinants, built from the single-particle basis
functions described in the previous section.

The principal idea of CI is to expand the wavefunction in the SD-basis
\begin{equation}
 \hat{H} \sum_{p=1}^{N_{\text{sd}}} c_{p,k} \ket{\Phi_p} = E_k \sum_{p=1}^{N_{\text{sd}}} c_{p,k} \ket{\Phi_p}.
\end{equation}
If we left-project with $\bra{\Phi_q}$ the above equation can be written on matrix form as 
\begin{equation}
 HC = CE
\end{equation}
where $H_{qp} \equiv \braket{\Phi_q|\hat{H}|\Phi_p}$, $E$ is a
diagonal matrix with the eigenvalues along its diagonal and $C$ is a
matrix containing the expansion coefficients of the $k$-th
eigenfunction in column $k$. Thus if we can compute the matrix
elements $H_{qp}$ the TISE can be solved by numerical diagonalization.

The evaluation of $H_{qp}$ is simplified by what is known as the
Slater-Condon rules. If $\ket{\Phi}$ is any SD, the only non-zero
contributions to $H$ are given by
\begin{align}
 \braket{\Phi| \left(\hat{H}_1 + \hat{H}_2\right)|\Phi} &= \sum_i \braket{i|\hat{h}_0|i} + \frac{1}{2}\sum_{ij}\braket{ij|\hat{u}|ij} \label{SlaterCondonEqual}\\
 \braket{\Phi| \left(\hat{H}_1 + \hat{H}_2\right)|\Phi_k^c} &= \braket{k|\hat{h}_0|c} + \sum_{i} \braket{ki|\hat{u}|ci} \label{SlaterCondonDiffByOne} \\
 \braket{\Phi| \left(\hat{H}_1 + \hat{H}_2\right)|\Phi_{kl}^{cd}} &= \braket{kl|\hat{u}|cd} \label{SlaterCondonDiffByTwo}.
\end{align}
In other words, we only have to consider the overlap between
determinants that differ by at most two occupied single-particle
functions.  Here we use the convention that $i,j,k,l = 1,...,N$, with
$N$ being the number of electrons and $a,b,c,d = N+1,...$ and $p,q,r,s
= 1,...,L$. Thus, we see that if the matrix elements
$\braket{p|\hat{h}_0|q}$ and $\braket{pq|\hat{u}|rs}$ are known we can
set up the Hamilton matrix and diagonalize it, obtaining the expansion
coefficients $C$.

Ideally we would like $N_{\text{sd}} \rightarrow \infty$ which is
impossible in practice. If we choose a finite number, $L$, of
single-particle basis functions and use all the linearly independent
Slater determinants we can build from this basis we get what is known
as the Full Configuration Interaction (FCI). However, the problem with
FCI is that the number of linearly independent determinants are given
by
\begin{equation}
 N_{\text{sd}} = \binom{L}{N}
\end{equation}
with $N$ being the number of electrons. This means that the numbers of
SD's becomes huge very fast, making FCI practical/feasible only for
the smallest systems. On the other hand, FCI is exact within a given
computational space, thus it is an invaluable tool in aiding us
determining the validity of more sophisticated methods. Furthermore,
CI is variational in the sense that it guarantees an upper bound on
the groundstate energy.

One way to adress the problems with FCI is to consider a
\textit{reference} determinant $\ket{\Phi}$, typically taken to be the
Hartree-Fock state, and consider excitations upon this state. That is,
we make the ansatz
\begin{equation}
 \ket{\Psi} = A_0 \ket{\Phi} + \sum_{ia} A_{ia} \ket{\Phi_{i}^a} + \frac{1}{4} \sum_{ijab} A_{ijab} \ket{\Phi_{ij}^{ab}} + \cdots
\end{equation}
where $\ket{\Phi_i^a}$ is a singly-excited determinant,
$\ket{\Phi_{ij}^{ab}}$ doubly-excited etc. and truncate at some
excitation level.  This results in a hierarchy of approximations known
as CIS (only single-excitations), CID (only double-excitations), CISD
(singles and doubles and so on. In this way one obtains polynomial
scaling of the number of determinants. The problem with this approach
is that the truncated CI-wavefunction is not size-consistent and
extensive \cite{Helgaker00book}. It turns out that this problem is
resolved by the Coupled Cluster method, which makes it the de facto
standard of many-body methods today.

\subsubsection{Time dependent Configuration interaction}
Another advantage with CI is that it is straightforward to generalize
to time dependent problems. Recall that the time evolution of an
arbitrary state, $\ket{\Psi(t)}$ with initial condition
$\ket{\Psi(t_0)}$, is given by the time dependent Schrödinger
equation,
\begin{equation}
 i \hbar \frac{\partial}{\partial t}\ket{\Psi(t)} = \hat{H}(t) \ket{\Psi(t)}.
\end{equation}
We will take the initial condition to be some (normalized) linear
combination of the CI eigenfunctions. In particular, we are interested
in the time evolution of the groundstate.

The time-dependent Configuration Interaction (TDCI) method is obtained
by placing a time-dependence on the expansion coefficients in the
CI-wavefunction while keeping the single-particle basis
fixed. Analogous to what we did in the previous section, we expand
$\ket{\Psi(t)}$ in a given orthonormal Slater determinant basis $\{
\Phi_p\}_{p=1}^{N_{\text{sd}}}$
\begin{equation}
 \ket{\Psi(t)} = \sum_{p=1}^{N_{\text{sd}}} A_p(t) \ket{\Phi_p}.
\end{equation}
If we now insert this expansion into the time-dependent Schrödinger equation we obtain,
\begin{equation}
 i \hbar  \sum_{p=1}^{N_{\text{sd}}} \frac{\partial A_p(t)}{\partial t} \ket{\Phi_p} =  \sum_{p=1}^{N_{\text{sd}}} A_p(t) \hat{H}(t) \ket{\Phi_p}. 
\end{equation}
Left-projecting the above equation with $\bra{\Phi_q}$ and using the
orthonormality of the Slater determinants, we are left with a
differential equation for each expansion coefficient $A_J(t)$ which
can be written as the matrix-vector equation
\begin{equation}
 i\hbar \frac{\partial A(t)}{\partial t} = H(t)A(t). \label{TDFCIEquation}
\end{equation}
Here $H(t)$ is the time dependent Hamilton matrix defined in terms of its elements,
\begin{equation}
 H_{qp}(t) = \braket{\Phi_q|\hat{H}(t)|\Phi_p}
\end{equation}
and $A(t) = [A_0(t), A_1(t), \cdots]$ is the vector containing the
expansion coefficients. We refer to this as the time dependent
Configuration Interaction (TDCI) method.

Completely analogous to the time independent case we have a hierarchy
of TDCI methods defined by the Slater determinant basis. Thus we have
a series of possible approximations ranging from TD-CIS to TD-FCI.  If
we did not have to truncate the Slater determinant basis, TDFCI would
be an exact solution to the TDSE.  However, we must in practice
truncate the basis and TDFCI suffers from the fact that the size of
FCI-space grows exponentially.  In order to obtain a good
approximation to the exact solution we would need a huge fixed
basis\cite{Kvaal12}.

On the other hand it is quite straightforward to implement, especially
when a program for computing the CI-groundstate already has been
written.  Again, the fact that the CI and CC space are equivalent for
$N=2$ makes it worthwhile to write a TDCI program, which we can use to
compare with the time dependent Coupled Cluster method which we
discuss in the next chapter.
\begin{comment}
\subsection{Coupled cluster theory}
In this section we review Coupled Cluster (CC) theory.  We will look
at the "classical" way of obtaining the CC equations. The presentation
borrows heavily from the excellent review by Crawford and
Schaefer\cite{Crawford2000}.

Furthermore we look at a different approach which in addition to the
usual CC equations also gives a method to solve the time dependent
Schrödinger equation, the so-called Orbital Adaptive Coupled Cluster
(OATDCC) method\cite{Kvaal12}.

\subsubsection{The exponential ansatz}
As usual we start by considering the time-independent Schrödinger equation
\begin{equation*}
\hat{H} \ket{\Psi} = E\ket{\Psi}.
\end{equation*}
In Coupled Cluster theory an exponential ansatz 
\begin{equation}
\ket{\Psi}_{CC} \equiv e^{\hat{T}} \ket{\Phi} \label{ExponantialAnsatzCC},
\end{equation}
is used to approximate the exact solution. Here $\ket{\Phi}$ is a
reference Slater determinant and $\hat{T} = \sum_{i}^n \hat{T}_i$ is
the sum over n-orbital cluster operators.

The the one- and two-orbital cluster operators are defined as
\begin{equation}
\hat{T}_1 \equiv \sum_{ia} t_i^a c_a^\dagger c_i
\end{equation}
and 
\begin{equation}
\hat{T}_2 \equiv \frac{1}{4} \sum_{ijab} t_{ij}^{ab} c_b^\dagger c_j c_a^\dagger c_i.
\end{equation}
Generally an n-orbital cluster operator is defined as 
\begin{equation}
\hat{T}_n \equiv \left(\frac{1}{n!}\right)^2 \sum_{ijk...abc...}t_{ijk...}^{abc...} ...c_c^\dagger c_k c_b^\dagger c_j c_a^\dagger   c_i.
\end{equation}
%We see that the cluster operators are linear combinations of excitation operators as defined by \eqref{DefinitionOfExciationOperator}. 
Recalling the exponential series we can write,
\begin{equation}
 \ket{\Psi}_{CC} = e^{\hat{T}} \ket{\Phi} = \left( 1+\sum_{k=1}^\infty \frac{1}{k!}\hat{T}^k \right) \ket{\Phi}. \label{CCansatzSeriesExpansion}
\end{equation}

\subsection{The Coupled Cluster Equations}
We now proceed by inserting expression \eqref{ExponantialAnsatzCC}
into the time-independent Schrödinger equation
\begin{equation}
\hat{H}e^{\hat{T}} \ket{\Phi} = Ee^{\hat{T}} \ket{\Phi} \label{ExponantialAnsatzCCInsertedInSE}.
\end{equation}
Left-projecting with $\bra{\Phi}$ we get an expression for the energy,
\begin{align*}
 \braket{\Phi|\hat{H}e^{\hat{T}}|\Phi} &= E \braket{\Phi|e^{\hat{T}}|\Phi} = E \braket{\Phi|\Psi_{CC}} \\
				       &= E
\end{align*}
where intermediate normalization, $\braket{\Phi|\Psi_{CC}} = 1$, is
assumed. Furthermore one can obtain an expression for the amplitude
$t^{ab...}_{ij...}$ upon left-projection with
$\bra{\Phi^{ab...}_{ij...}}$,
\begin{equation}
 \braket{\Phi^{ab...}_{ij...}|\hat{H}e^{\hat{T}}|\Phi} = E\braket{\Phi^{ab...}_{ij...}|e^{\hat{T}}|\Phi}.
\end{equation}
Notice that this equations are non-linear (due to the presence of
$e^{\hat{T}}$) and depend on the energy equation. This is impractical
and is resolved by performing a similarity transformation of the
Hamiltonian.

Multiply Eq. \eqref{ExponantialAnsatzCCInsertedInSE} with
$e^{-\hat{T}}$ and left-project with $\bra{\Phi}$ in order to obtain
\begin{equation}
 \braket{\Phi|e^{-\hat{T}} \hat{H} e^{\hat{T}} | \Phi} = E \braket{\Phi|e^{-\hat{T}}e^{\hat{T}}|\Phi} = E. \label{SimTransEnergyEquation}
\end{equation}
Left-projection with $\bra{\Phi^{ab...}_{ij...}}$ now results in,
\begin{equation}
 \braket{\Phi^{ab...}_{ij...}|e^{-\hat{T}}\hat{H}e^{\hat{T}}|\Phi} = E \braket{\Phi^{ab...}_{ij...}|\Phi} = 0. \label{SimTransAmplitudeEquation}
\end{equation}
Equations \eqref{SimTransEnergyEquation} and
\eqref{SimTransAmplitudeEquation} are commonly referred to as the
Coupled Cluster equations. Notice now that the equation for the
amplitudes does not depend on the energy. The amplitude equation is
non-linear and has to be solved iteratively. We say that we have a
self-consistent solution to CC-equations if the energy converges to
some pre-defined precision.

In order to implement the Coupled Cluster method we need algebraic
expressions for the CC-equations.
\begin{theorem}[Baker-Campbell-Hausdorff expansion]
For any matrices $A$ and $S$ the Baker-Campbell-Hausdorff (BCH) expansion is given by,
\begin{equation}
e^{-S}Ae^{S} = A + [A,S] + \frac{1}{2}[[A,S],S] + \frac{1}{3!}[[[A,S],S],S] + ... \label{BCH},
\end{equation}
where 
$$[A,S] \equiv AS - SA$$ 
is the commutator between $A$ and $S$. 
\end{theorem}
Under the assumption that $\hat{H}$ contains at most a two-body
operator the BCH-expansion truncates after the first five terms on the
right-hand side of eq. \eqref{BCH}\cite{Crawfordtruncates}.  This fact
greatly simplifies the evaluation of the left-hand sides of the
equations \eqref{SimTransEnergyEquation} and
\eqref{SimTransAmplitudeEquation}.

These expressions can be evaluated by hand using the second
quantization formalism as demonstrated in\cite{Crawford2000} or by
diagrammatic techniques\cite{ShavittBartlett}. Alternatively one can
use the second quantization toolbox in the Python library
Sympy\footnote{\href{http://docs.sympy.org/latest/modules/physics/secondquant.html}{http://docs.sympy.org/latest/modules/physics/secondquant.html}}
to compute these expressions.
 
\subsection{A variational CC theory?}
One drawback with the CC-method is that it does not satisfy the
variational principle \eqref{variationalTheorem}. Consider the
operator $e^{-\hat{T}}\hat{H}e^{T}$ which appears in the similarity
transformed energy equation \eqref{SimTransEnergyEquation}. This
operator is not Hermitian. In order to see this compute
$\hat{T}_1^\dagger$,
\begin{equation}
 \hat{T}_1^\dagger = \left(\sum_{ia} t_i^a c_a^\dagger c_i \right) = \sum_{ia} (t_i^a)^* c_i^\dagger c_a \neq \hat{T}_1.
\end{equation}
Then it follows that
\begin{equation}
 \left(e^{-\hat{T}}\hat{H}e^{\hat{T}}\right)^\dagger = (e^{\hat{T}})^\dagger \hat{H} (e^{-\hat{T}})^\dagger = e^{\hat{T}^\dagger} \hat{H} e^{-\hat{T}^\dagger} \neq e^{-\hat{T}}\hat{H}e^{\hat{T}}.
\end{equation}
Hermiticity of the operator was a necessary condition for the
variational theorem, thus Eq. \eqref{SimTransEnergyEquation} is not
variational.  However, if $\hat{T}$ is not truncated the spectrum of
$e^{-\hat{T}}\hat{H}e^{T}$ is identical to the original Hermitian
operator $\hat{H}$, which justifies its use in quantum mechanical
models\cite{CrawfordNonVar}.

There has been attempts to construct a variational solution by
deriving the amplitude equations by minimizing the functional,
\begin{equation}
 \frac{\braket{\Psi|\hat{H}|\Psi}}{\braket{\Psi|\Psi}} = \frac{\braket{\Phi|(e^{\hat{T}})^\dagger \hat{H}e^{T}|\Phi}}{\braket{\Phi|(e^{\hat{T}})^\dagger e^{\hat{T}}|\Phi}}.
\end{equation}
Notice then that $(e^{\hat{T}})^\dagger \hat{H}e^{T}$ does not conform to the Baker-Campbell-Hausdorff formula and the series has no natural truncation, complicating 
matters greatly. 
\subsection{The Orbital Adaptive Time Dependent Coupled Cluster Method}
In the article\cite{Kvaal12} Simen Kvaal demonstrates that it is
possible to derive equations of motion for the CC-ansatz by computing
critical points of a bivariational expectation value functional.  In
his approach both the orbitals and CC-amplitudes are allowed to vary
and the method can be used to solve the time-dependent Schrödinger
equation.  The method is called the Orbital Adaptive Time-Dependent
Coupled Cluster (OATDCC) and is a hierarchical approximation to
MCTDHF. Furthermore, if the orbitals are held fixed in time the method
gives a approximation to the Time-Dependent CI method which we refer
to as the Time-Dependent Coupled Cluster (TDCC) method.  While TDFCI
and MCTDHF suffers from exponential scaling, OATDCC and TDCC achieves
polynomial scaling.

The main goal of this article is to implement the TDCC method for a
possibly time-dependent Hamiltonian.  As such we will in the following
summarize the key ingredients needed to achieve this.  For
mathematical details the interested reader is referred to the article
by Kvaal.

\subsubsection{The Bivariational Principle}
As we have established, the Coupled Cluster method is not variational
in the usual sense since the similarity transformed Hamiltonian,
$$ \bar{H} = e^{-\hat{T}}\hat{H}e^{\hat{T}} $$ is not Hermitian. Now
following Kvaal, let $A$ be an operator (possibly non-Hermitian) over
Hilbert space $\mathcal{H}$ and consider the expectation value
functional
\begin{equation}
 \mathcal{E}_A: \mathcal{H}^{\prime}\times \mathcal{H} \rightarrow \mathbb{C}, \ \ \mathcal{E}_A [\bra{\Psi^{\prime}}, \ket{\Psi}] = \frac{\braket{\Psi^{\prime} |A| \Psi} }{\braket{\Psi^{\prime}|\Psi}}. 
\end{equation}
This is referred to as the the bivariational expectation value
functional. In contrast to the usual variational principle
$\bra{\Psi^{\prime}}$ and $\ket{\Psi}$ are treated as independent
parameters. The conditions for $\delta \mathcal{E}_A = 0$, for all
independent variations of $\bra{\Psi^{\prime}}$ and $\ket{\Psi}$ is
\begin{equation}
 (A-a)\ket{\Psi} = 0 \ \ \text{and} \ \ \bra{\Psi^{\prime}}(A-a) = 0,
\end{equation}
with $$a = \mathcal{E}_A[\bra{\Psi^{\prime}},\ket{\Psi}] $$
being the value of $\mathcal{E}_A$ at the critical point. 

Equations of motion can be derived from so-called time-dependent
variational principles. One such variational is set up using the
Lagrange formulation\cite{BECK20001},
\begin{equation}
 \delta \int_0^T L[\ket{\Psi}]dt = 0,
\end{equation}
where the Lagrangian functional is given by 
\begin{equation}
L[\ket{\Psi}] = \braket{\Psi(t)|\hat{H}-i\hbar \frac{\partial}{\partial t}|\Psi(t)}
\end{equation}
with the boundary conditions $\delta L(t_1) = \delta L(t_2) = 0$. The functional
\begin{equation}
\mathcal{S}[\ket{\Psi}] \equiv \int_0^T \braket{\Psi(t)|\hat{H}-i\hbar \frac{\partial}{\partial t}|\Psi(t)} dt \label{ActionFunctional1}
\end{equation}
is referred to as the action functional and computing equations of
motion from $\delta \mathcal{S} = 0$ is known as the prinicipal of
least action.

By considering a bivariational generalization of the action functional
\eqref{ActionFunctional1},
\begin{align}
 \mathcal{S}[\bra{\Psi^{\prime}}, \ket{\Psi}] \equiv &\int_0^T \frac{\braket{\Psi^{\prime}(t)|i\hbar \frac{\partial}{\partial t}-\hat{H}|\Psi(t)}}{\braket{\Psi^{\prime}(t)|\Psi(t)}} dt \\
 &\int_0^T i\hbar \frac{\braket{\Psi^{\prime}(t)|\frac{\partial}{\partial t}\Psi(t)}}{\braket{\Psi^{\prime}(t)|\Psi(t)}} - \mathcal{E}_H[\bra{\Psi^{\prime}(t)}, \ket{\Psi(t)}] dt. \label{ActionFunctional}
\end{align}
Kvaal derives equations of motions generalizing the Coupled Cluster
method to the time domain.  Here, one should note the appearance of
the bivariational generalization of the energy expectation value,
\begin{equation}
 \mathcal{E}_H[\bra{\Psi^{\prime}}, \ket{\Psi}] = \frac{\braket{\Psi^{\prime} |\hat{H}| \Psi} }{\braket{\Psi^{\prime}|\Psi}} \label{BivarEnergy},
\end{equation}
which is required to be complex analytic as shown in Ref.\cite{Kvaal13}. 

\subsubsection{The Coupled Cluster Ansatz}
The fundamental idea in\cite{Kvaal12} is to introduce different
exponential parametrizations $\ket{\Psi}$, $\bra{\Psi^\prime}$ and
compute $\delta \mathcal{S} = 0$, with $\mathcal{S}$ given by
\eqref{ActionFunctional}, which ultimately results in time dependent
Coupled Cluster equations.

In particular one make the ansatzes 
\begin{align}
 \ket{\Psi} &= e^{\hat{T}}\ket{\phi} \\ 
 \bra{\Psi^{\prime}} &= \bra{\tilde{\phi}}e^{\hat{T}^\prime},
\end{align}
where $\hat{T}$ is the familiar excitation operator
\begin{equation}
 \hat{T} = \sum_{ia} \tau_i^a c_a^\dagger \tilde{c}_i + \frac{1}{2!^2}
 \sum_{ijab} \tau_{ij}^{ab} c_a^\dagger \tilde{c}_i c_b^\dagger
 \tilde{c}_j + \cdots
\end{equation}
and $\hat{T}^{\prime}$ is a de-excitation operator on the form
\begin{equation}
 \hat{T}^{\prime} = \sum_{ia} (\tau^{\prime})_a^i c_i^\dagger \tilde{c}_a + \frac{1}{2!^2} \sum_{ijab} (\tau^{\prime})_{ab}^{ij} c_i^\dagger \tilde{c}_a c_j^\dagger \tilde{c}_b + \cdots.
\end{equation}
Here $\ket{\phi} = \ket{\varphi_{p_1},\cdots,\varphi_{p_N}}$ and $\bra{\tilde{\phi}} = \bra{\tilde{\varphi}_{p_1},\cdots,\tilde{\varphi}_{p_N}}$ are Slater determinants 
built from biorthogonal single-particle functions, $\braket{\tilde{\varphi}_p|\varphi_q} = \delta_{pq}$,  belonging to separate 
Hilbert spaces $\mathcal{V}$ and $\tilde{\mathcal{V}}$. The biorthogonality condition implies the anticommutator 
relation 
\begin{equation}
 \{ \tilde{c}_p, c_q^\dagger \} \equiv \tilde{c}_p c_q^\dagger + c_q^\dagger \tilde{c}_p = \delta_{pq},
\end{equation}
which preserves Wicks theorem. Here $\tilde{c}_p, \tilde{c}_p^\dagger, c_p$ and $c_p^\dagger$ are creation and annihilation opeartotrs associated with 
the biorthogonal single-particle functions. It should be noted that when $\tilde{c}_p$ or $\tilde{c}_p^\dagger$ acts on the determinant $\ket{\phi}$ it 
is only responsible for removing or adding $\varphi_p$ (not $\tilde{\varphi}_p$) from the determinant\cite{Kvaal12}. The notation 
$\Phi = (\varphi_1,\cdots,\varphi_L)$ and $\tilde{\Phi} = (\tilde{\varphi}_1,\cdots,\tilde{\varphi}_L)$ to denote the set of orbitals used to build 
the Slater determinants $\ket{\phi}$ and $\bra{\tilde{\phi}}$.

In standard CC theory and virtually every other many-body method, the single-particle functions 
are taken to be orthonogal such that $\mathcal{V} \equiv \tilde{\mathcal{V}}^\prime$. This relaxation of
orthonormality of the single-particle functions is necessary to ensure that the bivariational functional is
complex analytic if the single-particle functions are to be treated as variational parameters\cite{Kvaal12}. 


Next, a change of variables from $(T^{\prime},T)$ to $(\Lambda, T)$, with
\begin{equation}
\Lambda = \sum_{ia} \lambda_a^i c_i^\dagger \tilde{c}_a + \frac{1}{2!^2} \sum_{ijab} \lambda_{ab}^{ij} c_i^\dagger \tilde{c}_a c_j^\dagger \tilde{c}_b + \cdots 
\end{equation} 
a de-excitation operator is performed such that
\begin{equation}
 \bra{\tilde{\Psi}} = \frac{\bra{\Psi^{\prime}}}{\braket{\Psi^{\prime}|\Psi}} = \bra{\tilde{\phi}}(1 + \Lambda)e^{-T}.
\end{equation}
Using this parametrization $\braket{\tilde{\Psi}|\Psi} = 1$. The amplitudes $\{\tau_i^a,\tau_{ij}^{ab},\cdots \}$ are the usual Coupled Cluster amplitudes
which we refer to as the excitation amplitudes or $\tau$-amplitudes. 
Also we have introduced a second set of amplitudes $\{ \lambda_a^i, \lambda_{ab}^{ij},\cdots \}$ 
which we will refer to as the de-excitation amplitudes or $\lambda$-amplitudes.

Now the bivariational expectation value functional \eqref{BivarEnergy} reads 
\begin{equation}
 \mathcal{E}_H [\lambda, \tau, \tilde{\Phi}, \Phi] = \braket{\tilde{\phi}|(I+\Lambda)e^{-T}\hat{H}e^{T}|\phi}.
\end{equation}
Inserting the wavefunction ansatzes into the action functional \eqref{ActionFunctional} one arrives at the following expression,
\begin{equation}
 \mathcal{S}[\lambda,\tau,\tilde{\Phi},\Phi] = \int_0^T i\hbar \braket{\tilde{\phi}|(I+\Lambda)e^{-T}\frac{\partial}{\partial t} e^{T}|\phi} - \mathcal{E}_H[\lambda(t),\tau(t),\tilde{\Phi}(t), \Phi(t)] dt.
\end{equation}
Kvaal then shows that the action can be written,
\begin{align}
 \mathcal{S}[\lambda,\tau,\tilde{\Phi},\Phi] &= \int_0^T i \hbar \sum_\mu \lambda_\mu \dot{\tau}^\mu - \mathcal{E}_{H-i\hbar D_0}[\lambda, \tau, \tilde{\Phi}, \Phi] dt \\ 
 &=\int_0^T i \hbar \lambda_\mu \dot{\tau}^\mu + \rho_p^q(h_q^p - i \hbar \eta_q^p) + \frac{1}{4} \rho_{pr}^{qs}u^{pr}_{qs} dt,
\end{align}
where the operator $D_0$ is defined as,
\begin{equation}
 D_0 \equiv \sum_{pq} \braket{\tilde{\varphi}_p|\dot{\varphi}_q}c_p^\dagger \tilde{c}_q.
\end{equation}
Here $\mu$ denotes a general excitation $\{ia, ijab, \cdots\}$. Furthermore,  
\begin{align*}
 \rho_p^q &= \rho_p^q(\lambda,\tau) \equiv \braket{\tilde{\Psi}|c_p^\dagger \tilde{c}_q|\Psi}, \\
 \rho_{pr}^{qs} &= \rho_{pr}^{qs}(\lambda,\tau) \equiv  \braket{\tilde{\Psi}|c_p^\dagger c_r^\dagger \tilde{c}_s \tilde{c}_q|\Psi}, \\
 h^p_q &= h^p_q(\tilde{\Phi}, \Phi) \equiv \braket{\tilde{\varphi}_p|h|\varphi_q}, \\
 \eta_q^p &= \eta_q^p(\tilde{\Phi}, \dot{\Phi}) \equiv \braket{\tilde{\varphi}_p|\dot{\varphi}_q}, \\
 u^{pr}_{qs} &= u^{pr}_{qs}(\tilde{\Phi},\Phi) \equiv \braket{\tilde{\varphi}_p \tilde{\varphi}_r|u|\varphi_q \varphi_s}_{AS}. \\
\end{align*}
The quantities $\rho_p^q$ and $\rho_{pr}^{qs}$ are the CC reduced one- and two-body density matrices. Note that these do not depend explicitly on the orbitals since 
they are evaluated by Wick's theorem which only depends on the anti-commutator relations \eqref{AntiCommutatorRelations}. Thus, they only depend on the amplitudes. 

\subsubsection{OATDCC equations}
Finally, equations of motion are derived by demanding $\delta \mathcal{S}[\bra{\tilde{\Psi}},\ket{\Psi}] = 0$ for all independent variations 
of $\bra{\tilde{\Psi}}$ and $\ket{\Psi}$. In general both the single-particle functions and the amplitudes are allowed to vary resulting in equations of motion 
for both the amplitudes and the single-particle functions. A detailed derivation can be found in the supplementary material to\cite{Kvaal12}.

For the amplitudes the equations of motion read,
\begin{align}
 i \hbar \dot{\tau}^\mu &= \frac{\partial}{\partial \lambda_\mu } \mathcal{E}_{H-i\hbar D_0} [\lambda, \tau, \tilde{\Phi}, \Phi ] = \braket{\tilde{\phi}_\mu | e^{-T} \left(H-i \hbar D_0\right) e^T | \phi} \label{TAmplitudesEquation}, \\
 -i \hbar \dot{\lambda}_\mu &= \frac{\partial}{\partial \tau^\mu } \mathcal{E}_{H-i\hbar D_0} [\lambda, \tau, \tilde{\Phi}, \Phi ] = \braket{ \tilde{\phi} | (1+\Lambda)e^{-T} [H-i \hbar D_0,X_\mu] e^T | \phi} \label{LAmplitudesEquation}.
\end{align}
Here $X_\mu$ are the excitation operators, 
\begin{align*}
 X_i^a &= c_a^\dagger \tilde{c}_i, \\
 X_{ij}^{ab} &= c_a^\dagger \tilde{c}_i c_b^\dagger \tilde{c}_j, \\ 
 &\vdots \\
\end{align*}
Note that in the absence of $D_0$ the right-hand side of \eqref{TAmplitudesEquation} is equivalent with the usual amplitude equations \eqref{SimTransAmplitudeEquation}.

The equations for the unknowns $\eta_q^p(\tilde{\Phi}, \dot{\Phi})$ are given by
\begin{align}
 i\hbar \sum_{bj} A_{aj}^{ib} \eta_b^j &=   \sum_p \rho_p^i h_a^j - \sum_q \rho_a^q h_b^i + \frac{1}{2} \left[ \sum_{prs} \rho_{pr}^{is}u_{as}^{pr} - \sum_{rqs} \rho_{ar}^{qs}u_{qs}^{ir} \right] \\
 -i\hbar \sum_{bj} A_{bi}^{ja} \eta_j^b &=  \sum_p \rho_p^a h_i^b -\sum_q \rho_i^q h_j^a + \frac{1}{2} \left[ \sum_{prs} \rho_{pr}^{as}u_{is}^{pr} - \sum_{rqs} \rho_{ir}^{qs}u_{qs}^{ar} \right] + i \hbar \dot{\rho}_i^a,
\end{align}
where 
\begin{equation*}
 A_{aj}^{ib} \equiv \langle \tilde{\Psi} | [c_j^\dagger \tilde{c}_b, c_a^\dagger \tilde{c}_i] | \Psi \rangle = \delta_a^b \rho_j^i - \delta_j^i \rho_a^b. 
\end{equation*}
For the single-particle functions the equations of motion are given by 
\begin{align}
  i\hbar \sum_q \rho_p^q Q \ket{\dot{\varphi}_q} &= \sum_q \rho_p^q Q h \ket{\varphi_q} + \sum_{qrs} \rho_{pr}^{qs} Q W_s^r \ket{\varphi_q} \label{QspaceketEq}\\
 -i\hbar \sum_p \rho_p^q\bra{\dot{\tilde{\varphi}}_p}Q &= \sum_p \rho_p^q\bra{\tilde{\varphi}_p}hQ + \sum_{prs} \rho_{pr}^{qs} \bra{\tilde{\varphi}_p}W_s^rQ,  \label{QspacebraEq}
\end{align}
where $Q = I-P$ with $P$ being the projection operator
\begin{equation}
 \Phi \tilde{\Phi} \equiv \sum_p \ket{\varphi_p}\bra{\tilde{\varphi}_p}.
\end{equation}
The operators $W^r_s$ are defined by,
\begin{equation}
 W^r_s \ket{\psi} \equiv \braket{\cdot \tilde{\varphi}_r|u|\psi \varphi_s} \equiv \int dx_1 \int dx_2 \tilde{\varphi}_r(x_2) u(x_1,x_2) \psi(x_1) \varphi_s(x_2) 
\end{equation}
Equations \eqref{TAmplitudesEquation}-\eqref{QspacebraEq} is collectively referred to as the OATDCC equations. 

It is not at all obvious how one should compute the 
initial state for the OATDCC equations when the single-particle functions are allowed to vary 
and Kvaal proposes several different solutions in his article. 
\subsubsection{The TDCC equations}
A special case of OATDCC arises if the orbitals are held fixed in time. Then the operator $D_0$ drops and the OATDCC equations is reduced to only the amplitude 
equations
\begin{align}
 i \hbar \dot{\tau}^\mu &= \frac{\partial}{\partial \lambda_\mu } \mathcal{E}_{H} [\lambda, \tau, \tilde{\Phi}, \Phi ] = \langle \tilde{\phi}_\mu | e^{-T} H e^T | \phi \rangle \label{TAmplitudesEquationTDCC} \\
 -i \hbar \dot{\lambda}_\mu &= \frac{\partial}{\partial \tau^\mu } \mathcal{E}_{H} [\lambda, \tau, \tilde{\Phi}, \Phi ] = \langle \tilde{\phi} | (1+\Lambda)e^{-T} [H,X_\mu] e^T | \phi \rangle \label{LAmplitudesEquationTDCC},
\end{align}
which we refer to as the TDCC-equations. Notice now that equations \eqref{TAmplitudesEquationTDCC} and \eqref{SimTransAmplitudeEquation} are equivalent. Furthermore, 
we take $\tilde{\Phi} = \Phi^*$.

The initial amplitudes $(\lambda^{(0)},\tau^{(0)})$ are obtained by solving the non-linear equations
\begin{align}
 \langle \tilde{\phi}_\mu | e^{-T} H e^T | \phi \rangle &= 0\label{TAmplitudesEquationTDCCGS} \\
 \langle \tilde{\phi} | (1+\Lambda)e^{-T} [H,X_\mu] e^T | \phi \rangle &= 0 \label{LAmplitudesEquationTDCCGS}.
\end{align}

For the time evolution one then propgates the amplitudes by solving equations \eqref{TAmplitudesEquationTDCC} and \eqref{LAmplitudesEquationTDCC}. 
The energy is taken as the real part of
\begin{equation}
 \mathcal{E}_H [\lambda(t), \tau(t)] = \braket{\tilde{\phi}|(I+\Lambda)e^{-T}\hat{H}e^{T}|\phi} \label{CCexpecval},
\end{equation}
where the imaginary part should be "small". 

Approximations to the TDCC equations are obtained by truncating $T$ and $\Lambda$. For example 
$T = T_1 + T_2$ and $\Lambda = \Lambda_1 + \Lambda_2$ corresponds to the TDCCSD approximation while $T = T_2$ and $\Lambda = \Lambda_2$ 
gives the TDCCD approximation. 

It is shown in \cite{Kvaal12} that the singles amplitudes are redundant when the single-particle functions are varied. We will therefore restrict ourselves to the 
TDCCD approximation, using the Hartree-Fock state as reference determinant. This allows for flexibility if one wants to extend the program written in this work 
to the full OATDCC method.


\subsubsection{The TDCCD approximation} \label{TDCCD approximation section}
In order to write a computer program that solves the TDCC equations we need algebraic expressions for the right-hand sides of equations \eqref{TAmplitudesEquationTDCC}
and \eqref{LAmplitudesEquationTDCC}. Now, in the TDCCD approximation we take $T = T_2$ and $\Lambda = \Lambda_2$. 
We want expressions for the amplitudes and the one- and two-body density matrices. Thus we must evaluate the vacuum expectation values,
\begin{align}
 \frac{\partial}{\partial \lambda_{ab}^{ij} } \mathcal{E}_{H} &= \braket{\phi_{ij}^{ab}|e^{-T_2}He^{T_2}|\phi} \label{ExpecTamplitude}\\
 \frac{\partial}{\partial \tau^{ab}_{ij} } \mathcal{E}_{H} &= \braket{\phi|(1+\Lambda_2)e^{-T_2}[H, X_{ij}^{ab}]e^{T_2}|\phi}, \label{ExpexLamplitude}\\
 \rho_p^q &= \braket{\tilde{\Psi}|c_p^\dagger \tilde{c}_q|\Psi}, \\
 \rho_{pr}^{qs} &=  \braket{\tilde{\Psi}|c_p^\dagger c_r^\dagger \tilde{c}_s \tilde{c}_q|\Psi}.
\end{align}
These are generated by the \textit{CoupledClusterSympy.py} script, which uses the second quantization package in SymPy, 
and is found in the src folder at the github page\footnote{\href{https://github.com/haakoek/PythonVersionMaster/tree/master/src}{https://github.com/haakoek/PythonVersionMaster/tree/master/src}}. 
The expressions for the amplitude equations and one- and two-body density matrices are listed in the appendices.

Finally, the energy is taken as the real part of the expectation value \eqref{CCexpecval} which in the CCD approximation is given by,
\begin{align*}
 \mathcal{E}_H [\lambda(t), \tau(t)] &= \braket{\tilde{\phi}|(I+\Lambda_2)e^{-T_2}\hat{H}e^{T_2}|\phi} \\
				     &= \braket{\tilde{\phi}|e^{-T_2}\hat{H}e^{T_2}|\phi} + \braket{\tilde{\phi}|\Lambda_2e^{-T_2}\hat{H}e^{T_2}|\phi} \\
				     &= \braket{\tilde{\phi}|e^{-T_2}\hat{H}e^{T_2}|\phi} + \frac{1}{4} \sum_{ijab} \lambda_{ij}^{ab}\braket{\tilde{\phi}_{ij}^{ab}|e^{-T_2}\hat{H}e^{T_2}|\phi}
\end{align*}
where 
\begin{equation}
 \braket{\tilde{\phi}|e^{-T_2}\hat{H}e^{T_2}|\phi} = h_i^i + \frac{1}{2}u_{ij}^{ij} + \frac{1}{4}\tau_{ij}^{ab}u_{ab}^{ij}. \label{ECCDGroundstate}
\end{equation}
Notice that in the groundstate,
\begin{equation*}
 \braket{\tilde{\phi}_{ij}^{ab}|e^{-T_2}\hat{H}e^{T_2}|\phi} = 0
\end{equation*}
which means that the energy for the groundstate is simply given by \eqref{ECCDGroundstate}. 

Note also that the one- and two-body density matrices depend only on the amplitudes $(\tau,\lambda)$. This is an advantage when we want to verify the implementation of 
the $\lambda$-amplitudes since we can compare the density matrices with the density matrices computed with the CI method. 

In order to verify that the $\tau$-amplitudes has been computed correctly for the groundstate 
we compute the groundstate energy and compare with previous published results.


\section{Computational details}

\section{Results}

\section{Conclusions}

\begin{appendices}
 \section{Properties of Cluster operators.}
\begin{itemize}
\item $\hat{T}_{N+1} \ket{\Phi} = 0$, where $\hat{T} = \hat{T}_1 + \hat{T}_2 + \cdots$, $\ket{\Phi}$ a reference determinant and $N$ the number of particles 
in $\ket{\Phi}$. This implies that for a given $N$ the cluster operator $\hat{T}$ has a finite number of terms,
\begin{equation}
 \hat{T} = \hat{T}_1 + \hat{T}_2 + \cdots + \hat{T}_N.
\end{equation}
 \item The rank of 
an excitation/cluster operator is defined as the number of creation or annihilation operators, i.e rank$\left(\hat{T}_2\right) = 2$.
\item The product of two exciation/cluster operators is an exciation/cluster operator.
\item All excitation/cluster operators commute. We can prove this by considering the action of the commutator on a reference determinant as follows,
\begin{align*}
 [c_b^\dagger c_j, c_a^\dagger c_i] \ket{\Phi} &= \left(c_b^\dagger c_j c_a^\dagger c_i - c_a^\dagger c_i c_b^\dagger c_j \right) \ket{\Phi} \\
 &= \ket{\Phi_{ij}^{ab}} \ket{\Phi_{ji}^{ba}} \\
 &= \ket{\Phi_{ij}^{ab}} - \ket{\Phi_{ij}^{ab}} \\
 &= 0 \cdot \ket{\Phi_{ij}^{ab}}.
\end{align*}
Thus $c_b^\dagger c_j c_a^\dagger c_i = c_a^\dagger c_i c_b^\dagger c_j$, meaning that $c_b^\dagger c_j$ and $c_a^\dagger c_i$ commute. The result for a 
general excitation operator follows by induction.
\item rank$\left(\hat{T}_n \hat{T}_m \right) = n+m$.
\item Excitation operators are nilpotent, 
\begin{equation}
 \hat{\tau}_X^2 = \left(\cdots c_c^\dagger c_k c_b^\dagger c_j c_a^\dagger c_i \right)^2 = 0.
\end{equation}
This follows immediately from the fact that, 
\begin{align*}
 \left(c_a^\dagger c_i\right)^2 \ket{\Phi} = c_a^\dagger c_i \ket{\Phi_i^a} = 0 
\end{align*}
since $c_i \ket{\Phi_i^a} = 0$.
\item $\hat{\tau}_X^\dagger = \left(\cdots c_c^\dagger c_k c_b^\dagger c_j c_a^\dagger c_i \right)^\dagger = \left(c_i^\dagger c_a c_j^\dagger c_b c_k^\dagger c_c \cdots\right) $ 
is a de-excitation operator. Thus,
\begin{equation}
 \bra{\Phi}\hat{\tau}_X = \hat{\tau}^\dagger_X \ket{\Phi} \equiv 0.
\end{equation}
\item Since all cluster operators commute, $e^{\hat{T}+\hat{T}^\prime} = e^{\hat{T}} e^{\hat{T}^\prime}$.
\end{itemize}

\section{Existence of cluster expansion.}
Suppose $\braket{\Phi|\Psi} = 1$. Then there exists cluster operators $\hat{T}$ and $\hat{A}$ such that 
\begin{equation}
 \ket{\Psi} = \left( 1 + \hat{A} \right) \ket{\Phi} = e^{\hat{T}} \ket{\Phi}.
\end{equation}
\subparagraph*{Proof:} Let $X$ denote a general excitation index and let $\tau_X$ be a general excitation operator. Assume that 
$\braket{\Phi|\Psi} = 1$ and let $$\{ \ket{\Phi}, \ket{\Phi_X} | \text{ for all } X \}$$ be a Slater determinant basis where $\braket{\Phi|\Phi_X} = 0$. We can expand $\ket{\Psi}$ in the basis, giving
\begin{equation}
 \ket{\Psi} = A_0 \ket{\Phi} + \sum_X A_X \ket{\Phi_X}, \ \ A_0, A_X \in \mathbb{C}.
\end{equation}
Note that $\braket{\Phi|\Psi} = 1$ implies that $A_0 = 1$. Using $\ket{\Phi_X} = \hat{\tau}_X \ket{\Phi}$ we get, 
\begin{equation}
 \ket{\Psi} = \ket{\Phi} + \sum_X A_X \hat{\tau}_X \ket{\Phi} = \left(1 + \hat{A}\right) \ket{\Phi},
\end{equation}
where we have defined $\hat{A} \equiv \sum_X A_X \hat{\tau}_X$. We now want to find $\hat{T}$ such that $e^{\hat{T}} = 1 + \hat{A}$. Rewrite $\hat{A}$ 
as,
\begin{align*}
 \hat{A} = \sum_X A_X \hat{\tau}_X &= \sum_{ia} A_i^a \hat{\tau}_i^a + \frac{1}{2!^2}\sum_{ijab} A_{ij}^{ab} \hat{\tau}_{ij}^{ab} + \cdots \\
 &= \hat{A}_1 + \hat{A}_2 + \cdots
\end{align*}
By expanding $e^{\hat{T}}$ as a series
\begin{equation}
 e^{\hat{T}} = 1 + \hat{T} + \frac{1}{2} \hat{T}^2 + \cdots \label{ExponantialExpansion},
\end{equation}
and note that $e^{\hat{T}} = 1 + \hat{A}$ if they have the same singles, doubles, etc. parts. Let $N$ be given and insert $\hat{T} = \hat{T}_1 + \cdots + \hat{T}_N$
into eq. \eqref{ExponantialExpansion}, expand each power and group terms of equal rank together,
\begin{align*}
 e^{\hat{T}} &= 1 + \hat{T} + \frac{1}{2} \hat{T}^2 + \frac{1}{6} \hat{T}^3 + \cdots \\
 &= 1 + \left(\hat{T}_1 + \cdots + \hat{T}_N \right) + \frac{1}{2} \left( \hat{T}_1 + \cdots + \hat{T}_N \right)^2 \\
 &+ \frac{1}{6} \left( \hat{T}_1 + \cdots + \hat{T}_N \right)^3 + \frac{1}{24} \left( \hat{T}_1 + \cdots + \hat{T}_N \right)^4 + \cdots \\
 &= 1 + \underbrace{\hat{T}_1}_{\text{rank} 1} + \underbrace{\left( \hat{T}_2 + \frac{1}{2}\hat{T}_1^2 \right)}_{\text{rank} 2} + \underbrace{\left( \hat{T}_3 + \frac{1}{6} \hat{T}_1^3 + \hat{T}_1 \hat{T}_2 \right)}_{\text{rank} 3} + \cdots \\
\end{align*}
Note that in the last equation, the rank $k$ term contains only one term from $\hat{T}_k$ while all other terms are products of cluster operators of lower rank.
For the above expression to be equal to $1 + \hat{A}_1 + \hat{A}_2 + \cdots$ each rank $k$ term must equal $\hat{A}_k$.
\begin{align*}
 \hat{A}_1 = &\hat{T}_1 \\
 \hat{A}_2 = &\hat{T}_2 + \frac{1}{2}\hat{T}_1^2 \\
 \hat{A}_3 = &\hat{T}_3 + \frac{1}{6} \hat{T}_1^3 + \hat{T}_1 \hat{T}_2 \\
  &\vdots
\end{align*}
Then we can find $\hat{T}_k$ recursively, 
\begin{align}
 \hat{T}_1 = &\hat{A}_1 \\
 \hat{T}_2 = &\hat{A}_2 - \frac{1}{2}\hat{T}_1^2 = \hat{A}_2 - \frac{1}{2}\hat{A}_1^2  \\
 \hat{T}_3 = &\hat{A}_3 - \frac{1}{6} \hat{T}_1^3 - \hat{T}_1 \hat{T}_2 \\
           = &\hat{A}_3 - \frac{1}{6} \hat{A}_1^3 - \hat{A}_1 \left( \hat{A}_2 - \frac{1}{2}\hat{A}_1^2 \right) \\
 & \vdots
\end{align}
Hence, given $\ket{\Psi}$, we can find $\hat{A}$ which in turn gives $\hat{T}$.

We notice that the expansion $\ket{\Psi} = (1+\hat{A}) \ket{\Phi}$ actually is the CI expansion. Thus, the 
CC expansion $e^{\hat{T}}\ket{\Phi}$ is a nonlinear reparametrization of the linear CI parametrization. In practice, for a system containing $N$ particles, 
one must truncate $\hat{T} = \hat{T}_1 + \cdots + \hat{T}_N$ and $\hat{A} = \hat{A}_1 + \cdots + \hat{A}_N$. Now, truncate the expansion at the 
$k$-th operator in the sum with $1 \leq k \leq N$ such that $\hat{T} = \hat{T}_1 + \cdots + \hat{T}_k$ and $\hat{A} = \hat{A}_1 + \cdots + \hat{A}_k$. 
Observe that 
\begin{align*}
\ket{\Psi} &= (1+\hat{A}_1 + \cdots + \hat{A}_k) \ket{\Phi} \\
\end{align*}
is a linear combination where the highest excitation contribution has rank $k$. The exponential ansatz on the other hand 
\begin{align*}
 e^{\hat{T}_1 + \cdots + \hat{T}_k} \ket{\Phi} = \left( 1 + \left( \hat{T}_1 + \cdots + \hat{T}_k \right) + \frac{1}{2} \left( \hat{T}_1 + \cdots + \hat{T}_k \right)^2 + \cdots \right) \ket{\Phi}
\end{align*}
contains contributions from determinants of higher excitation than rank $k$ which is immediately clear from the inclusion of the term $\hat{T}_k^2$ 
in the second parenthesis in the above expression. It turns out that this gives the truncated CC ansatz an advantage over 
the truncated CI ansatz. The latter is not size-consistent and extensive while the former is. 

Loosely speaking size-consistency and extensivity 
means that if $A$ and $B$ are two noninteracting subsystems, we should get the same energy for the supersystem $AB$ irrespective of whether we have
carried out the calculations for each subsystem separately or for both subsystems simultaneously. In general 
truncated exponential wavefunctions have this property while truncated linear wavefunctions does not\cite{Helgaker00book}. 

Depending on the truncation we get a sequence of more and more
accurate CC approximations:
\begin{itemize}
 \item $\ket{\Psi_{CCS}} = e^{\hat{T}_1} \ket{\Phi}$ coupled-cluster singles (CCS)
 \item $\ket{\Psi_{CCD}} = e^{\hat{T}_2} \ket{\Phi}$ coupled-cluster doubles (CCD)
 \item $\ket{\Psi_{CCSD}} = e^{\hat{T}_1 + \hat{T}_2} \ket{\Phi}$ coupled-cluster singles doubles (CCSD)
 \item $\ket{\Psi_{CCSDT}} = e^{\hat{T}_1 + \hat{T}_2 + \hat{T}_3} \ket{\Phi}$ coupled-cluster singles doubles triples (CCSDT)
\end{itemize}

\section{Equality of CC and CI for $N=2$.}

An important special case is the case $N=2$. In that case the CC
expansion equals the CI expansion. This is invaluable information when
we want to implement the CC method since we can then compare it with
the corresponding CI method to verify the implementation. To see this
let $N=2$ and let $\ket{\Phi}$ be a reference determinant as
usual. Observe that $\hat{A}_3 \ket{\Phi} = 0 $ and $\hat{T}_3 =
0$. Thus, $\hat{A} = \hat{A}_1 + \hat{A}_2$ and $\hat{T} = \hat{T}_1 +
\hat{T}_2$ will give all non-zero contributions. Additionaly all
products of cluster operators where the product has rank larger than
$2$, acting on the reference will give a zero result.  Inserting
$\hat{T}_1 = \hat{A}_1$ and $\hat{T}_2 = \hat{A}_2 -
\frac{1}{2}\hat{A}_1^2$ into the CC expansionincluding only terms of
at most rank $2$ we obtain,
 \begin{align*}
  e^{\hat{T}_1 + \hat{T}_2} \ket{\Phi} &= \left( 1 +
  \underbrace{\hat{T}_1}_{\text{rank} 1} + \underbrace{\left(
    \hat{T}_2 + \frac{1}{2}\hat{T}_1^2 \right)}_{\text{rank} 2}
  \right) \ket{\Phi} \\ &= \left(1+ \hat{A}_1 + \hat{A}_2 -
  \frac{1}{2}\hat{A}_1^2 + \frac{1}{2} \hat{A}_1^2 \right) \ket{\Phi}
  \\ &= \left(1+ \hat{A}_1 + \hat{A}_2 \right) \ket{\Phi}
 \end{align*}
which indeed is the CI expansion. For $N=2$, $\hat{A} = \hat{A}_1 +
\hat{A}_2$ corresponds to a full configuration interaction treatment,
while $\hat{T} = \hat{T}_1 + \hat{T}_2$ CCSD, but in this case CCSD is
actually equal to FCI. Morover, if we consider only $\hat{T} =
\hat{T}_2$ (CCD), we get
\begin{align*}
 \ket{\Psi}_{\text{CCD}} = e^{\hat{T}_2} \ket{\Phi} &= \left(1+
 \hat{T}_2 \right) \ket{\Phi} \\ &= \left( 1 + \hat{A}_2 -
 \frac{1}{2}\hat{A}_1^2 + \frac{1}{2} \hat{A}_1^2 \right) \ket{\Phi}
 \\ &= \left(1 + \hat{A}_2 \right) \ket{\Phi} =
 \ket{\Psi}_{\text{CID}}.
\end{align*}
The importance of these two results is that one can verify a CCD
implementation by comparing with a CID implementation and a CCSD
implementation can be verified by comparing with a FCI/CISD
implementation.

\section{Amplitude equations}
Recall that the Hamiltonian in second quantization is given by,
\begin{equation}
 \hat{H} = \hat{H}^{(1)} + \hat{H}^{(2)} = \sum_{pq} h^p_q c_p^\dagger c_q + \frac{1}{4} \sum_{pqrs} u^{pr}_{qs} c_p^\dagger c_r^\dagger c_s c_q
\end{equation}
where $ u^{pr}_{qs} = \braket{\varphi_p
  \varphi_r|\hat{H}^{(2)}|\varphi_q \varphi_s}_{AS}$.  Evaluation of
the right hand side of equation \eqref{ExpecTamplitude} gives,
\begin{equation}
 \frac{\partial}{\partial \lambda_{ij}^{ab}}E_{H^{(1)}} = h^{k}_{i}
 \tau^{ab}_{jk} P(ij) + h^{a}_{c} \tau^{bc}_{ji}
 P(ab) \label{AlgebraicTampH1},
\end{equation}
\begin{align}
 \frac{\partial}{\partial \lambda_{ab}^{ij}}E_{H^{(2)}} =
 &-\tau^{ab}_{ik} u^{kl}_{jl} P(ij) + \tau^{ab}_{il} \chi_j^l P(ij)
 \nonumber \\ &+\tau^{ab}_{kl} \chi^{kl}_{ij} +\tau^{ac}_{ij}\chi_c^b
 P(ab) \nonumber \\ &+\tau^{ac}_{ik} \chi_{cj}^{kb}P(ab)P(ij)
 -\frac{1}{2}\tau^{dc}_{ji} u^{ab}_{dc} \nonumber
 \\ &+u^{ab}_{ij} \label{AlgebraicTampH2},
\end{align}
where $P(ij)$ is an anti-symmtrizer in the sense that $f(ij)P(ij) =
f(ij)-f(ji)$ and similarly for $P(ab)$. We have defined the so-called
intermediates,
\begin{align}
 \chi^{kl}_{ij} &= \frac{1}{4} \tau^{cd}_{ij} u^{kl}_{cd} + \frac{1}{2} u^{kl}_{ij} \label{Xklij} \\
 \chi_c^b &= \frac{1}{2} \tau^{bd}_{kl}  u^{kl}_{dc}  +  u^{bk}_{ck} \label{Xcb} \\
 \chi_j^l &= \frac{1}{2}\tau^{dc}_{jk} u^{kl}_{dc} \label{Xjl} \\
 \chi_{cj}^{kb} &= u^{kb}_{cj} + \frac{1}{2} u^{kl}_{cd}\tau^{bd}_{jl} \label{Xcjkb} .
\end{align}
We use the Einstein summation convention over repeated indices of
opposite vertical placement. The appearance of a $P(ij)$ or a $P(ab)$
should be ignored for the invocation of the summation convention. This
means that for example the expression $h^{k}_{i} \tau^{ab}_{jk} P(ij)$
is interpreted as,
\begin{equation*}
 h^{k}_{i} \tau^{ab}_{jk} P(ij) = \sum_k \left(h^{k}_{i} \tau^{ab}_{jk} - h^{k}_{j} \tau^{ab}_{ik} \right).
\end{equation*}
In other words, we sum over all indices that does not appear on the
left-hand side of all following expressions.

Evaluation of the right hand side of equation \eqref{ExpexLamplitude}
results in,
\begin{align}
 \frac{\partial}{\partial \tau_{ij}^{ab}}E_{H^{(1)}} = h^{c}_{a} \lambda^{ji}_{bc} P(ab) + h^{i}_{k} \lambda^{jk}_{ab} P(ij) \label{AlgebraicLampH1},
\end{align}
\begin{align}
 \frac{\partial}{\partial \tau_{ij}^{ab}}E_{H^{(2)}} = &\lambda^{kl}_{ac} \xi^{cji}_{klb} P(ab) +\lambda^{ik}_{dc} \xi^{dcj}_{kab}  P(ij) \nonumber \\
 +&\lambda^{kl}_{ab}\xi_{kl}^{ij} +\lambda^{ik}_{ac}\xi_{kb}^{cj} P(ab) P(ij) \nonumber \\
 +&\lambda^{ik}_{ab}\xi_k^j P(ij) +\lambda^{ji}_{dc}\xi_{ab}^{dc} \nonumber \\
 +&\lambda^{ji}_{ac} \xi_b^c P(ab) -u^{ji}_{ab} \label{AlgebraicLampH2},
\end{align}
where we have defined the intermediates,
\begin{align}
 \xi^{cji}_{klb}      &= - \frac{1}{2}\tau^{dc}_{kl} u^{ji}_{bd} \label{Ecjiklb} \\
 \xi^{dcj}_{kab}            &= - \frac{1}{2}\tau^{dc}_{kl} u^{jl}_{ab} \label{Edcjkab} \\
 \xi_{kl}^{ij} &= -\frac{1}{2} u^{ji}_{kl}  - \frac{1}{4} \tau^{dc}_{kl}u^{ji}_{dc} \label{Eklij} \\
 \xi_{kb}^{cj} &= - \tau^{dc}_{kl} u^{jl}_{bd} +  u^{jc}_{bk} \\
 \xi_k^j &= -\frac{1}{2} \tau^{dc}_{kl} u^{jl}_{dc}  -  u^{jl}_{kl} \\
 \xi_{ab}^{dc} &= -\frac{1}{4}\tau^{dc}_{kl}u^{kl}_{ab} - \frac{1}{2}u^{dc}_{ab} \\
 \xi_b^c &= -\frac{1}{2} \tau^{dc}_{kl} u^{kl}_{bd} - u^{ck}_{bk} \label{Ebc}.
\end{align}
If we precompute and store all intermediates equation \eqref{AlgebraicTampH2} scale as 
$$\mathcal{O}\left(N^2(L-N)^4\right)$$ 
while 
equation \eqref{AlgebraicLampH2} scale as
$$\mathcal{O}\left(N^3(L-N)^4\right).$$
\section{Density matrices}
For the onebody density matrix we get
\begin{align}
 \rho_i^j &= \delta_{i j} - \frac{1}{2}\lambda^{kj}_{ab} \tau^{ab}_{ki}, \label{RHOij} \\
 \rho_a^b &= \frac{1}{2}\lambda^{ij}_{ac} \tau^{bc}_{ij}, \\
 \rho_i^a &= \rho_a^i = 0.
\end{align}
The non-zero elements of the two-body density matrix are given by,
\begin{align}
 \rho_{ij}^{kl} &= - \delta_{i l} \delta_{j k} P(ij) - \frac{1}{2}P(ij) P(kl)\delta_{j l}\lambda^{km}_{dc} \tau^{dc}_{im}   + \frac{1}{2}\lambda^{lk}_{dc} \tau^{dc}_{ji}, \\
 \rho_{ab}^{ij} &= \lambda^{ij}_{ab} \\
 \rho_{ia}^{jb} &= \frac{1}{2}\delta_{i j} \lambda^{kl}_{ac} \tau^{bc}_{kl}  - \lambda^{jk}_{ac} \tau^{bc}_{ik} = -\rho_{ia}^{bj} = -\rho_{ai}^{jb} = \rho_{ai}^{bj} \\
 \rho_{ab}^{cd} &= \frac{1}{2}\lambda^{ij}_{ab} \tau^{cd}_{ij} \\
 \rho_{ij}^{ab} &= - \frac{1}{4} \lambda^{kl}_{dc} \tau^{ab}_{ik} \tau^{dc}_{jl} P(ab) P(ij) - \frac{1}{4} \lambda^{kl}_{dc} \tau^{ab}_{kl} \tau^{dc}_{ji} - \frac{1}{4} \lambda^{kl}_{dc} \tau^{ac}_{kl} \tau^{bd}_{ji} P(ij) \nonumber \\
 &+ \lambda^{kl}_{dc} \tau^{ac}_{il} \tau^{bd}_{jk} P(ij) - \frac{1}{4} \lambda^{kl}_{dc} \tau^{ac}_{ji} \tau^{bd}_{kl} P(ij) - \tau^{ab}_{ji} \label{RHOijab}.
\end{align}

\end{appendices}
\end{comment}
\bibliographystyle{ieeetr}
\bibliography{references}

\end{document}


