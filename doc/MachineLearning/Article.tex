\documentclass[aip,jcp,reprint,floatfix]{revtex4-1}

\usepackage{amsmath,amssymb}
\usepackage{epsfig}
\usepackage{graphicx}
\usepackage{dcolumn}
\usepackage{bbold}
\usepackage{natbib}
\usepackage[normalem]{ulem} % to strike out plain text
\usepackage[colorlinks=true,citecolor={blue},urlcolor={blue}, linkcolor=blue]{hyperref}%
\usepackage[utf8]{inputenc}
\usepackage{amsthm}
\usepackage{braket}
\usepackage{verbatim}
\usepackage{mdframed}
\newtheorem{theorem}{Theorem}


\begin{document}


\title{Machine Learning and the Many-body Problem}

\author{Jane Kim}
\affiliation{National Superconducting Cyclotron Laboratory and Department of Physics and Astronomy, Michigan State University, East Lansing, MI 48824, USA}
\affiliation{Department of Physics, University of Oslo, N-0316 Oslo, Norway}

\author{Vilde Flugsrud}
\affiliation{Department of Physics, University of Oslo, N-0316 Oslo, Norway}

\author{Alocias Madarson}
\affiliation{Department of Physics, University of Oslo, N-0316 Oslo, Norway}


\author{Morten Hjorth-Jensen}
\affiliation{National Superconducting Cyclotron Laboratory and Department of Physics and Astronomy, Michigan State University, East Lansing, MI 48824, USA}
\affiliation{Department of Physics and Center for Computing in Science Education, University of Oslo, N-0316 Oslo, Norway}



\begin{abstract}

\end{abstract}

\pacs{32.80.Rm, 31.15.A-, 42.65.Ky}

\maketitle




\section{Introduction}

\bibliographystyle{ieeetr}
\bibliography{references}

\end{document}


