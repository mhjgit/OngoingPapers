\documentclass[reprint,floatfix]{revtex4-1}


\usepackage{graphicx}
\usepackage{comment}
\usepackage{lipsum}
\usepackage[utf8]{inputenc}
\usepackage[toc,page]{appendix}
\usepackage{braket}
\usepackage{amsmath,amssymb}
\usepackage{amsthm}
\usepackage{hyperref}
\usepackage{pgfplots}
\usepackage{bbold}



\begin{document}

\title{Integrating a Computational Perspective in Physics Education}

\author{Marcos Daniel Caballero}
\affiliation{Department of Physics and Astronomy, Michigan State University, East Lansing, Michigan 48824, USA}
\affiliation{Department of Physics and Center for Computing in Science Education, University of Oslo, N-0316 Oslo, Norway}
\author{Larry P. Engelhardt}
\affiliation{Department of Physics and Engineering, Francis Marion University, Florence, South Carolina 29502, USA}
\author{Morten Hjorth-Jensen}
\affiliation{National Superconducting Cyclotron Laboratory and Department of Physics and Astronomy, Michigan State University, East Lansing, Michigan 48824, USA}
\affiliation{Department of Physics and Center for Computing in Science Education, University of Oslo, N-0316 Oslo, Norway}
\author{Marie Lopez del Puerto}
\affiliation{Department of Physics, University of St. Thomas, St. Paul, Minnesota 55105, USA}
\author{Kelly Roos}
\affiliation{Department of Mechanical Engineering, Bradley University, Peoria, Illinois 61625, USA}




\date{\today}




\begin{abstract}

Demands on computing in research continue to grow.  How can we
integrate computing in core Physics courses in a coordinated and
coherent way?  
Computing competence represents a central element in
scientific problem solving, from basic education and research to
essentially almost all advanced problems in modern societies.  

\end{abstract}

\maketitle

\section{Introduction}


Computational modeling and data analysis is core to a variety of
non-academic career paths that are often pursued by physics students
following the receipt of their Bachelor's degree.


Computational modeling and data analysis is core to a variety of
non-academic career paths that are often pursued by physics students
following the receipt of their Bachelor's degree.

A 2012 AIP survey \cite{AIPsurvery2012} of 5,000 recent physics Bachelor’s degree recipients
indicated that 50$\%$ chose to enter the workforce immediately after
receiving their degree.  Of these, roughly 75$\%$ go into STEM-related
fields including engineering, software development, or information
technology.

Depending on sub-field, 60-80$\%$ of physics degree-holders in
industry regularly need to engage in computer programming, simulation
and modeling tasks.  Many respondents to the survey stressed the
importance of computer skills, including programming, in increasing
their marketability to potential employers.


The use of computational modeling in the classroom setting provides
students with insights that are complementary to those resulting from
pencil-and-paper manipulation of equations.

Computing in core physics courses allows us to bring important
elements of scientific methods at a much earlier stage in our
students' education.


The ability to closely examine the behavior of systems that are too
complex to be easily analytically tractable, or that have no analytic
solutions (i.e., many systems of practical interest), helps to develop
intuition that is unavailable to many students from analytic
calculation.

The growth in computer power over the past decades has radically
changed science and its applications. Computational modeling, using
computers and programming to solve, visualize, and explain phenomena, 
is now an integrated and central part of research, development, and
innovation. Yet, this change is sparsely reflected in most educations,
even in science and engineering. Thus, there is a need to renew
curricula and integrate computational modeling to ensure students are
prepared for the 21st century workplace, a development that opens
pedagogical challenges and opportunities. However, because the use of
computational modeling is sparse, there is a near absence of research
on students’ use of computational modeling in education. In addition,
there is a growing need for students at all lveles and across disciplines for
digital skills and computational literacy. This requires the infusion
of aspects of computational and computer science into educational
programs at all levels. Such a transformation of the contents of our
education calls for new content in teacher education, new curricula
both in computing and in disciplinary subjects with integrated
computing, and research-based methods for instruction at all
levels. The lack of such curricula, adequately trained teachers, and
the supporting education research is a significant national and
international challenge. In the US, this was recently addressed in a
report on Computer Science education research \cite{USreport}. Similarly, in the
European Union there is a significant focus on development of digital skills
\cite{EUreport}.  Both reports signal an urgent
need for curricular development, professional development programs,
and education research on computational methods and their integration
in various disciplines.


\bibliographystyle{ieeetr}


\bibliography{refs}
\end{document}


The case for improving U.S. Computer Science Education, see https://itif.org/publications/2016/05/31/case-improving-us-computer-science-education
European Commission, New  report shows digital skills are required in all types of jobs, see https://ec.europa.eu/digital-single-market/en/news/new-report-shows-digital-skills-are-required-all-types-jobs.
Association for Computing Machinery, 2013, see http://pathways.acm.org/executive-summary.html.
Partnership for Integration of Computation into Undergraduate Physics, see https://www.compadre.org/PICUP/
Marcos Daniel Caballero and Morten Hjorth-Jensen, Integrating a Computational Perspective in Physics Courses,  http://lanl.arxiv.org/abs/1802.08871, and Nova publishers, in press (2018); How to introduce computing in basic physics courses,  planned published as University Texts in Physics, Springer.
